\documentclass[17pt]{extarticle}
%\usepackage[paperheight=4in]{geometry}
\usepackage[top=1cm, bottom=1cm, left=2cm, right=2cm]{geometry}
\pagestyle{empty} %no page numbering
\usepackage[utf8]{inputenc}
\usepackage{graphicx}
\usepackage{amsmath}
\usepackage{amssymb}
\usepackage{amsthm}
\usepackage{listings}

\newtheorem{theorem}{Theorem}
\newtheorem{proposition}[theorem]{Proposition}
\newtheorem{lemma}[theorem]{Lemma}
\newtheorem{example}{Example}
\newtheorem*{example*}{Example}
\newtheorem{definition}{Definition}
\newtheorem*{definition*}{Definition}
\newtheorem{remark}[theorem]{Remark}
\newtheorem*{theorem*}{Theorem}
\newtheorem*{condition*}{Condition}
\newtheorem*{problem*}{Problem}
\newtheorem*{algorithm*}{Algorithm}

\setlength\parindent{0pt} %no indent

\begin{document}
First we aim to formulate the problem in mathematical terms.
For this we enumerate the $n$ vertices of the regular polyhedron in clockwise direction by the set
$$
S_n:= \{0, 1, \cdots, n-1\}
$$
Moreover we equip this set with the group structure of $\mathbb{Z}_n$, that is, 
$$a+b:=(a+b) \mod n$$
where the '$+$'-operation on the r.h.s is the one from $\mathbb{Z}$.
In words, adding $1$ to a vertex results in the next vertex in clockwise direction.\\ \\
For every pair $(i,j)$ with $i,j\in S_n$ there exists a unique $k<n$ such that $i+k=j$.
This can be used to split $S_n$ into two disjoint sets:
$$
A:=\{i+s \ : \ 0\leq s \leq k\}
$$
$$
B:=S_n \backslash A
$$
In the sequel these sets will be just referred as the $A$-set resp. $B$-set of $(i,j)$.
\\ \\
We call a pair $(i,j)\in S_n\times S_n$ a \textbf{diagonal}, if for some $1<k<n-1$ 
$i+k=j$. Such a pair corresponds to a diagonal in the considered regular polyhedron.
\\ \\
Let $(i,j)$ be a diagonal and $A, B$ the disjoint sets defined above.
Then a second diagonal $(k,l)$ is said to \textbf{intersect} with $(i,j)$ if either $k\in A \ \wedge \ l\in B$
or $l\in A \ \wedge \ k\in B$. One easily verifies that this definition exactly describes two diagonals
to intersect in the inner of the polyhedron.\\ \\
This definitions allow us to formulate the problem in mathematical terms:
\begin{problem*}
	For given $n,k\in\mathbb{Z}_+$ with $n>k$ find the number of all $k$-element sets of non-intersecting diagonals
	in $S_n$.
\end{problem*}
The following algorithm (pseudo-code) is solving the above problem:
\begin{algorithm*}
\begin{lstlisting}[language=Python]
	
res <- 0
for i in [0, ..., n-1] 
  for j in [i+2, ..., i-2]
    if k = 1 and j >= i // (*)
      res <- res + 1
    else
      res <- res 
        + count_sub_diags(k - 1, i, j, n, i)

      
def count_sub_diags(k, lower, upper, n, root)
  if upper + 1 = lower or upper + 2 = lower
    return 0
  
  if k = 1
    return 1

  res <- 0
  for i in [upper + 2, ..., lower - 1]
    if i >= root // (**)	
      res <- res 
        + count_sub_diags(k - 1, lower, i, n, root)

  return res
    
 \end{lstlisting}
\end{algorithm*}
Note, the intervals $[i, ..., j]$ used in the above algorithm are meant to present cyclic intervals.
So for instance if $n=5$, the interval $[3, ..., 2]$ presents the tuple $(3, 4, 0, 2)$.
In addition, operations on vertexes are to be interpreted as $\mathbb{Z}_n$-group operations.
So, for instance, $i+1$ in terms of integers means: $(i+1)\mod n$.\\ \\
The algorithm starts by using $i=0$ as the lower end of a diagonal and $j=2$ as its upper end.
In case of $k=1$, we found a $k$-element set of diagonals and we can increase the counter $res$.
Otherwise, we increase the counter by the number of $(k-1)$-element sets of diagonals found in the
$B$-set of $(i,j)$. This number is computed in the method $count\_sub\_diags$.\\
This is proceeded by first iterating $j$, that is, the upper end of the diagonal, until it reaches two vertexes 
before $i$, and then iterating $i$, that is the lower end of the diagonal.

\begin{theorem*}
	The above algorithm correctly returns the number of all $k$-element sets of diagonals.
\end{theorem*}
\begin{proof}
	The proof will be done by induction on $n$. It is easy to see, that the algorithm correctly works for the case $n\leq 4$. Next, we show that if it correctly works for $n-1$ then it will work for $n$ as well.\\
	To show the algorithm to collect all $k$-element sets of diagonals we consider a given set $S$ of $k$ non intersecting diagonals. Let $(i,j)\in S$ be any of these diagonals and $D$ then one within the cyclic interval $[i, i+1, \cdots, j]$ with starting point most next to $i$ and containing no diagonals of $S$ inside its $A$-set. It is clear such a diagonal to exist by the assumption all diagonals in $S$ to not intersect.
	Now, let the vertex $i$ in the first $for$-loop be the one corresponding to the lower end of $D$.
	Since all the remaining $k-1$ diagonals of $S$ are within $D$'s $B$-set and since the function $count\_sub\_diags$ is counting all such $k-1$-element sets in exactly this $B$-set, the algorithm will correctly count the set $S$.\\
	Next, we show that each $k$-element set of diagonals is counted at most ones. To the contrary, let us assume there is a set $S$ of $k$ diagonals that is counted twice by the algorithm. If $k=1$ this would mean some 
	$i_1, i_2$ from the first $for$-loop with $i2>i_1$ would produce the same diagonal. But this is impossible 
	by the condition that is marked with $(*)$ in the algorithm. The latter ensures to skip those diagonals that have been counted for by a previous vertex $i$.\\
	If $k>1$, then there must be two diagonals $(i_1, j_1)$ and $(i_2, j_2)$ produced at the first $for$-loops with $i1<i2$, that both generate $S$ in its later course. Necessarily, $(i_1, j_1)$ must lie in the $B$-set of $(i_2, j_2)$ and vice versa. By construction this implies $S$ is a loop. Moreover, $(i_1, j_1)$ must be reached from $(i_2, j_2)$ within the search for sub-diagonals, that is, in $count\_sub\_diags(.,.,.,n,i_2)$.
	But now the condition flagged with $(**)$ in the algorithm skips exactly this sub-diagonal, since the $root$ is $i_2$ and the vertex $i$ is $i_1$ (remember: $i1<i_2$ by assumption).
\end{proof}






\end{document}

	
