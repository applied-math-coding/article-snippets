\documentclass[17pt]{extarticle}
%\usepackage[paperheight=3in]{geometry}
\pagestyle{empty} %no page numbering
\usepackage[utf8]{inputenc}
\usepackage{graphicx}
\usepackage{amsmath}
\usepackage{amssymb}

\newtheorem{theorem}{Theorem}[section]
\newtheorem{proposition}[theorem]{Proposition}
\newtheorem{lemma}[theorem]{Lemma}
\newtheorem{example}[theorem]{Example}
\newtheorem{definition}[theorem]{Definition}
\newtheorem{remark}[theorem]{Remark}

\setlength\parindent{0pt} %no indent

\begin{document}
\begin{equation} \label{main_eq}
c-a = (1-q)(d-a)
\end{equation}
Which states that $c$ shall be located so that it is $q(d-a)$
left from $d$. In other words, $c$ shall be the point
next to the upper endpoint of the new interval.
From the construction we know two further equations:
\begin{equation*}
	c-a=q(b-a)
\end{equation*}
(in accordance how we have chosen $c$ in the first step)
\begin{equation*}
	d-a=(1-q)(b-a)
\end{equation*}
(in accordance how we have chosen $d$ in the first step)
Using this in (\ref{main_eq}) gives
$$
q(b-a)=(1-q)^2(b-a)
$$
or after canceling out $b-a$, just
$$
q=(1-q)^2
$$
This equation can be solved for $q$ and yields a solution
of
$$
q=\frac{3}{2}-\frac{\sqrt{5}}{2}
$$
lying within $(0,1/2)$.
\end{document}