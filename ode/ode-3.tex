\documentclass[17pt]{extarticle}
%\usepackage[paperheight=4in]{geometry}
\usepackage[top=2cm, bottom=2cm]{geometry}
\pagestyle{empty} %no page numbering
\usepackage[utf8]{inputenc}
\usepackage{graphicx}
\usepackage{amsmath}
\usepackage{amssymb}
\usepackage{amsthm}

\newtheorem{theorem}{Theorem}[section]
\newtheorem{proposition}[theorem]{Proposition}
\newtheorem{lemma}[theorem]{Lemma}
\newtheorem{example}[theorem]{Example}
\newtheorem{definition}[theorem]{Definition}
\newtheorem{remark}[theorem]{Remark}
\newtheorem*{theorem*}{Theorem}
\newtheorem*{condition*}{Condition}

\setlength\parindent{0pt} %no indent

\begin{document}
In order to ease notations, let us assume for a moment the function $f$ fulfills a Lipschitz-condition of the form
\begin{equation} \label{lipschitz}
|f(t, x)-f(t,y)|\leq L|x-y|
\end{equation}
for some $L>0$.
From our scheme we have
\begin{equation} \label{scheme}
	y_{n+1}=y_n+f(t_n, y_n)
\end{equation}
If $y$ denotes the true solution, then by Taylor we may write
\begin{equation*} 
	y(t_{n+1})=y(t_n)+y'(t_n)h+o(h^2)
\end{equation*}
Further using $y'(t)=f(t,y(t))$ this yields
\begin{equation} \label{true_y}
	y(t_{n+1})=y(t_n)+f(t_n, y(t_n))h+h^2R
\end{equation}
In the last step we have in addition replaced $o(h^2)$ by some estimate $R$ of the remainder.
Our aim is to compare $y_{n+1}$ with $y(t_{n+1})$, therefore let us subtract (\ref{true_y}) from (\ref{scheme}):
$$
y_{n+1}-y(t_{n+1})=y_n-y(t_n)+f(t_n, y_n)-f(t_n, y(t_n))h-h^2R
$$
By introducing $e_k:=y_{k}-y(t_k)$ we further can estimate
$$
|e_{n+1}|\leq |e_n|+Lh|e_n|+h^2R
$$
Note, we have used (\ref{lipschitz}).\\
 We reformulate appropriately to obtain
$$
|e_{n+1}|\leq (1+Lh)(|e_n|+hR/L)-hR/L
$$
which gives
$$
|e_{n+1}|+hR/L\leq (1+Lh)(|e_n|+hR/L)
$$
In order to ease computation we make the 'pessimistic' estimate of $1+Lh\leq exp(Lh)$.
Thus we have
\begin{equation} \label{main_inequ}
|e_{n+1}|+hR/L\leq exp(Lh)(|e_n|+hR/L)
\end{equation}
This holds for all indexes and so also
$$
|e_{n}|+hR/L\leq exp(Lh)(|e_{n-1}|+hR/L)
$$
Inserting the later into (\ref{main_inequ}) gives
$$
|e_{n+1}|+hR/L\leq exp(2Lh)(|e_{n-1}|+hR/L)
$$
By proceeding this way, we find
$$
|e_{n+1}|+hR/L\leq exp((n+1)Lh)(|e_{0}|+hR/L)
$$
Altogether this shows
\begin{theorem*}
$$
|e_{n+1}|\leq exp(L(b-a))(|e_{0}|+hR/L)-hR/L
$$
\end{theorem*}
One can finally weaken the assumption of a global Lipschitz-condition for $f$. Without giving all the details here, the idea is to replace it by:
\begin{condition*}
There exists $L, M>0$ such that
\begin{equation*} 
	|f(t, x)-f(t,y)|\leq L|x-y|
\end{equation*}
for all $|x-y|\leq M$.
\end{condition*}
This is like a local version of (\ref{lipschitz}). By carefully repeating above proof, one can see that  it is always possible to choose $h$ small enough in order to have $|y_n-y(t_n)|\leq M$. The rest of the argumentation remains the same.
\end{document}