\documentclass[17pt]{extarticle}
%\usepackage[paperheight=5in, top=1cm, bottom=1cm]{geometry}
\pagestyle{empty} %no page numbering
\usepackage[utf8]{inputenc}
\usepackage{graphicx}
\usepackage{amsmath}
\usepackage{amssymb}

\newtheorem{theorem}{Theorem}[section]
\newtheorem{proposition}[theorem]{Proposition}
\newtheorem{lemma}[theorem]{Lemma}
\newtheorem{example}[theorem]{Example}
\newtheorem{definition}[theorem]{Definition}
\newtheorem{remark}[theorem]{Remark}

\setlength\parindent{0pt} %no indent

\begin{document}
By assuming $y$ is 3-times differentiable we can look at two Taylor-expansions both around $t+h/2$. One for the point shifted by $h/2$,
\begin{equation} \label{taylor_one}
	y(t+h)=y\left(t+\frac{h}{2}\right)+\frac{h}{2}y'\left(t+\frac{h}{2}\right)+\frac{1}{8}y''\left(t+\frac{h}{2}\right)+o(h^3)
\end{equation}
and the other at the point shifted by $-h/2$:
\begin{equation} \label{taylor_two}
	y(t)=y\left(t+\frac{h}{2}\right)-\frac{h}{2}y'\left(t+\frac{h}{2}\right)+\frac{1}{8}y''\left(t+\frac{h}{2}\right)+o(h^3)
\end{equation}
Now we subtract (\ref{taylor_two}) from (\ref{taylor_one}) to obtain
\begin{equation*}
	y(t+h)-y(t)=hy'\left(t+\frac{h}{2}\right)+o(h^3)
\end{equation*}
and this shows
$$
\frac{y(t+h)-y(t)}{h}-y'\left(t+\frac{h}{2}\right)=o(h^3)
$$
This is our first local error. But we have to deal with another which arises from the approximation of $y(t+h/2)$. Simply by applying Taylor we see
$$
y\left(t+\frac{h}{2}\right)=y(t)+\frac{h}{2}y'(t)+o(h^2)
$$
which gives second local error estimate
$$
y\left(t+\frac{h}{2}\right)-y(t)-\frac{h}{2}f(t,y(t))=o(h^2)
$$

\end{document}