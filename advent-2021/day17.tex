\documentclass[17pt]{extarticle}
%\usepackage[paperheight=4in]{geometry}
\usepackage[top=1cm, bottom=1cm, left=2cm, right=2cm]{geometry}
\pagestyle{empty} %no page numbering
\usepackage[utf8]{inputenc}
\usepackage{graphicx}
\usepackage{amsmath}
\usepackage{amssymb}
\usepackage{amsthm}
\usepackage{listings}

\newtheorem{theorem}{Theorem}
\newtheorem{proposition}[theorem]{Proposition}
\newtheorem{lemma}[theorem]{Lemma}
\newtheorem{example}{Example}
\newtheorem*{example*}{Example}
\newtheorem{definition}{Definition}
\newtheorem*{definition*}{Definition}
\newtheorem{remark}[theorem]{Remark}
\newtheorem*{theorem*}{Theorem}
\newtheorem*{condition*}{Condition}
\newtheorem*{problem*}{Problem}
\newtheorem*{algorithm*}{Algorithm}

\setlength\parindent{0pt} %no indent

\begin{document}
The target area is given by
$$[\xi_1, \xi_2]\times [\eta_1, \eta_2]$$
If the initial velocity in horizontal resp. vertical direction is $u_0$ resp. $v_0$, then
after $n$ steps the horizontal resp. vertical position of the particle is
\begin{equation} \label{x_n}
x_n=n u_0-\frac{1}{2}n(n-1)
\end{equation}
resp.
\begin{equation} \label{y_n}
y_n=n v_0-\frac{1}{2}n(n-1)
\end{equation}
 
 For the vertical velocity $v_n$ at step $n>0$ we have
 $$v_n=v_{n-1}-1$$
 Thus,
 $$v_n=v_0-n$$
 When the particle reaches the highest point, it must necessarily fulfill
 $$v_m=0$$
 Therefore the previous equation implies
 $$m=v_0$$
 By putting this in the equation (\ref{y_n}) yields:
 $$y_{max}=\frac{1}{2}v_0(v_0+1)$$
 The latter shows, finding the highest possible $v_0$, automatically gives the highest possible
 vertical position $y_{max}$.\\ \\
 For given $u_0$ let us compute all possible steps that makes the particle horizontally ending up in $[\xi_1, \xi_2]$:\\
 By (\ref{x_n}) the determining relation for this is
 $$\xi_1\leq n u_0-\frac{1}{2}n(n-1)\leq \xi_2$$
 This is an quadratic relation in $n$. By using $\xi_1, \xi_2\geq 0$ one easily verifies 
 that the $n\geq 0$ that satisfy the above relation are exactly those that satisfy
 $$\sqrt{\left(u_0+\frac{1}{2}\right)^2-2\xi_2}+u_0+\frac{1}{2}\leq n\leq \sqrt{\left(u_0+\frac{1}{2}\right)^2-2\xi_1}+u_0+\frac{1}{2}$$
 \\
 Next, for give $u_0$ and given $n$ within the above interval, let us determine the interval of possible $v_0$ such that the particle vertically ends up within $[\eta_1, \eta_2]$ at step $n$.\\
 By use of (\ref{y_n}), we can formulate this as
 $$\eta_1\leq nv_0-\frac{1}{2}n(n-1)\leq \eta_2$$
 This yields
 $$\frac{1}{n}\left(\eta_1+\frac{1}{2}n(n-1)\right)\leq v_0 \leq \frac{1}{n}\left(\eta_2+\frac{1}{2}n(n-1)\right)$$
 \\
 Note, when doing the corresponding iterations, one has to ensure by use of suitable Gauss-brackets, all velocities and steps to be natural numbers. 
\end{document}

	
