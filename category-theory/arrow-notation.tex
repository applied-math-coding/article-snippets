\documentclass[17pt]{extarticle}
%\usepackage[paperheight=4in]{geometry}
\usepackage[top=1cm, bottom=1cm, left=2cm, right=2cm]{geometry}
\pagestyle{empty} %no page numbering
\usepackage[utf8]{inputenc}
\usepackage{graphicx}
\usepackage{amsmath}
\usepackage{amssymb}
\usepackage{amsthm}

\newtheorem{theorem}{Theorem}[section]
\newtheorem{proposition}[theorem]{Proposition}
\newtheorem{lemma}[theorem]{Lemma}
\newtheorem{example}{Example}
\newtheorem{definition}{Definition}
\newtheorem{remark}[theorem]{Remark}
\newtheorem*{theorem*}{Theorem}
\newtheorem*{condition*}{Condition}

\setlength\parindent{0pt} %no indent

\begin{document}
Nevertheless, the category in which objects are sets and morphisms functions between sets provides an important example. It is denoted by 
$$Set$$

For a morphism $f$ with $dom(f)=A$ and $cod(f)=B$ one often uses the notation
$$
f:A\rightarrow B
$$
But remember, although this even more resembles the notation of functions, $f$ is only in specific
cases a function. The only thing which morphisms are ensure to have in common with functions, is associative composition and a special morphism which behaves like identity (see its definition).\\ \\
The class of all morphisms is denoted by
$$Hom(A,B)$$
Likewise, the class of all morphisms in a category $C$ is denoted by
$$Hom(C)$$




\end{document}

	
