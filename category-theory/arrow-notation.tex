\documentclass[17pt]{extarticle}
%\usepackage[paperheight=4in]{geometry}
\usepackage[top=1cm, bottom=1cm, left=2cm, right=2cm]{geometry}
\pagestyle{empty} %no page numbering
\usepackage[utf8]{inputenc}
\usepackage{graphicx}
\usepackage{amsmath}
\usepackage{amssymb}
\usepackage{amsthm}

\newtheorem{theorem}{Theorem}[section]
\newtheorem{proposition}[theorem]{Proposition}
\newtheorem{lemma}[theorem]{Lemma}
\newtheorem{example}{Example}
\newtheorem{definition}{Definition}
\newtheorem{remark}[theorem]{Remark}
\newtheorem*{theorem*}{Theorem}
\newtheorem*{condition*}{Condition}

\setlength\parindent{0pt} %no indent

\begin{document}


This definition shall remind us to sets and functions, that is objects to sets and morphisms to
functions between those sets. Important to understand is, that the definition does not restrict to
this specific interpretation. Interpreting it as sets rather provides a specific example, the category of sets which we denote by $Set$. Trivially, for the latter all the above axioms are fulfilled.\\ \\

To ease notation one usually writes for a morphism $f$ with $dom(f)=A$ and $cod(f)=B$
$$
f:A\rightarrow B
$$
and denotes the set of all these morphisms by
$$Hom(A,B)$$
But again, if this reminds even more to functions between sets, we don't require $f$ to be a function nor
$A, B$ to be sets.\\ \\
So what else can it be if these are not functions?\\ \\
For this, we look at the following example of a class which fulfills all the above axioms and for which
morphisms are not functions.\\ \\



\end{document}

	
