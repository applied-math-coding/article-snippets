\documentclass[17pt]{extarticle}
%\usepackage[paperheight=4in]{geometry}
\usepackage[top=1cm, bottom=1cm, left=2cm, right=2cm]{geometry}
\pagestyle{empty} %no page numbering
\usepackage[utf8]{inputenc}
\usepackage{graphicx}
\usepackage{amsmath}
\usepackage{amssymb}
\usepackage{amsthm}

\newtheorem{theorem}{Theorem}
\newtheorem{proposition}[theorem]{Proposition}
\newtheorem{lemma}[theorem]{Lemma}
\newtheorem{example}{Example}
\newtheorem*{example*}{Example}
\newtheorem{definition}{Definition}
\newtheorem*{definition*}{Definition}
\newtheorem{remark}[theorem]{Remark}
\newtheorem*{theorem*}{Theorem}
\newtheorem*{condition*}{Condition}

\setlength\parindent{0pt} %no indent

\begin{document}
\begin{definition*}
	\textbf{Product}\\
	In a category $C$ the product of two objects $A_1, A_2$ is an object
	denoted by $A_1\times A_2$ together with two morphisms
	$$p_1:A_1\times A_2\rightarrow A_1$$
	and
	$$p_2:A_1\times A_2\rightarrow A_2$$
	such that the following holds:\\
	For any object $X$ and morphisms $f_1:X\rightarrow A_1$ and $f_2:X\rightarrow A_2$ there exists a unique morphism denoted by $\langle f_1,f_2\rangle:X\rightarrow A_1\times A_2$ such that
	$$
	p_1\circ\langle f_1,f_2\rangle=f_1
	$$
	and
	$$
	p_2\circ\langle f_1,f_2\rangle=f_2
	$$
\end{definition*}
\leavevmode\newline
The morphisms $p_1, p_2$ are called \textbf{projections} and $\langle f_1, f_2\rangle$ the \textbf{product morphism} of $f_1, f_2$. This shall remind to the Cartesian product in the category $Set$ and one easily verifies that the Cartesian product indeed fulfills all the above stated requirements.\\
However note, for two object of an arbitrary category the product must not necessarily exist.
Before we proceed, let us emphasize the importance of the uniqueness of $\langle f_1, f_2 \rangle$ in the definition of a product. Shortly, we will see an example were exactly this uniqueness is the striking feature of products.
\\ \\
The notation of a product entails that the product itself only depends on the factors. It turns out that this is correct up to isomorphism.

\begin{theorem} \label{product_isomorphy}
	The product $A_1\times A_2$ of two objects $A_1, A_2$ is unique up to isomorphism.
\end{theorem}
\begin{proof}
	Let us assume there exists two objects $P$ and $Q$ with corresponding morphisms $p_1, p_2$ and $q_1, q_2$ such that both fulfill the requirements to be the product of $A_1$ and $A_2$. We show, there exists an isomorphism $j:P\rightarrow Q$:\\
	 We apply the product properties of $P$ onto the case $X=P$, $f_1=p_1$ and $f_2=p_2$. Then by assumption there is a unique morphism $i:P\rightarrow P$ such that $p_1\circ i = p_1$ and $p_2\circ i = p_2$. We conclude 
	\begin{equation} \label{i_equal_id_P}
	i=id_P
    \end{equation}
    The uniqueness played the central role in concluding the equation.\\
    We repeat this application but for the case $X=Q$, $f_1=q_1$ and $f_2=q_2$. This yields a unique morphism 
    $$\langle q_1,q_2 \rangle:Q\rightarrow P$$\\
    Next, in analogy, we consider the product properties of $Q$.
    Here we apply case $X=P$, $f_1=p_1$ and $f_2=p_2$ that implies the existence of  a morphism
    $$\langle p_1,p_2 \rangle:P\rightarrow Q$$\\
    We show that $\langle q_1,q_2 \rangle$ is the desirec isomorphism and $\langle p_1,p_2 \rangle$ its inverse:\\
     Using the properties of product morphism one gets:
    $$
    p_1\circ \langle q_1,q_2 \rangle \circ \langle p_1,p_2 \rangle = q_1\circ  \langle p_1,p_2 \rangle=p_1
    $$
    and
    $$
    p_2\circ \langle q_1,q_2 \rangle \circ \langle p_1,p_2 \rangle = q_2\circ  \langle p_1,p_2 \rangle=p_2
    $$
    Comparing this with the above morphism $i$ and its uniqueness, (\ref{i_equal_id_P}) implies
    $$
    \langle q_1,q_2 \rangle \circ \langle p_1,p_2 \rangle=id_P
    $$
    By analogous steps or just symmetry, one finds
    $$
    \langle p_1,p_2 \rangle \circ  \langle q_1,q_2 \rangle=id_P
    $$
    and we are done.	 
\end{proof}
The next theorem shows that the product is associative in case all factors do exist in $C$.

\begin{theorem}
	If $A_1, A_2, A_3$ are objects from a category $C$, then
	$$(A_1\times A_2)\times A_3=A_1\times (A_2\times A_3)$$
	under the condition that all factors are defined.
\end{theorem}
\begin{proof}
	The proof is similar to the one of theorem \ref{product_isomorphy}. By using
	several projections one yields a unique morphism $f:(A_1\times A_2)\times A_3\rightarrow A_1\times (A_2\times A_3)$ and another $g:A_1\times (A_2\times A_3)\rightarrow  (A_1\times A_2)\times A_3$. The like in proof of the above theorem, one compares the projections of $f\circ g$ and $g\circ f$ with the projections of $id:(A_1\times A_2)\times A_3\rightarrow (A_1\times A_2)\times A_3$ resp. $id:A_1\times (A_2\times A_3)\rightarrow A_1\times (A_2\times A_3)$.
\end{proof}
\leavevmode\newline
As long as all products exits, the last theorem can be used to define products of arbitrary many finite objects. So for instance 
$$A_1\times A_2\times A_3\times A_4=A_1\times (A_2\times A_3\times A_4)=
A_1\times (A_2\times (A_3\times A_4))$$
All this is independent from the specific way of nesting parentheses.\\
Also, from the above theorem we conclude, if a category contains the product of any two objects, then
it contains all finite products.


\end{document}

	
