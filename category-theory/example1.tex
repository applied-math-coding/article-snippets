\documentclass[17pt]{extarticle}
%\usepackage[paperheight=4in]{geometry}
\usepackage[top=1cm, bottom=1cm, left=2cm, right=2cm]{geometry}
\pagestyle{empty} %no page numbering
\usepackage[utf8]{inputenc}
\usepackage{graphicx}
\usepackage{amsmath}
\usepackage{amssymb}
\usepackage{amsthm}

\newtheorem{theorem}{Theorem}[section]
\newtheorem{proposition}[theorem]{Proposition}
\newtheorem{lemma}[theorem]{Lemma}
\newtheorem{example}{Example}
\newtheorem{definition}{Definition}
\newtheorem{remark}[theorem]{Remark}
\newtheorem*{theorem*}{Theorem}
\newtheorem*{condition*}{Condition}

\setlength\parindent{0pt} %no indent

\begin{document}
	

\begin{example}
	We take as objects all sets and as morphisms all binary relations between two sets.\\ \\
	 If $A$ and $B$ are sets then a binary relation is a subset $f\subset A\times B$. So if $(a,b)\in R$, then $a$ 'relates' to $b$.\\ \\
	 In general, such a relation is not a function. It could be that $(a,b)\in f$ and $(a, c)\in f$ for $b,c\in B$ with $b\neq c$.\\ 
	 
	We define composition between morphisms as follows: \\
    If $f:A\rightarrow B$ and $g:B\rightarrow C$ then
	$$
	(a,c)\in g\circ f \ :\leftrightarrow \ \exists b\in B \ (a,b)\in f \wedge (b,c)\in g 
	$$
	In words, the composition relates $a$ to $c$ if and only if there is a 'bridge' $b$ such that $f$ relates $a$ to $b$ and $g$ relates $b$ to $c$.\\
	
    For any object $A$ we define
    $$id_A =\{(a,a) \ | \ a\in A\}$$
    One immediately sees that this serves correctly as identity morphism.\\ \\
    It remains to proof associativity. Assume we are given $f:A\rightarrow B$, $g:B\rightarrow C$ and $h:C\rightarrow D$.
    $$
    (a,d)\in (h\circ g)\circ f
    $$
    means, there exists $b\in B$ such that
    \begin{equation} \label{a_b_f}
    (a,b)\in f
    \end{equation}
    and
    $$(b,d)\in h\circ g$$
    Moreover, the latter implies there exists $c\in C$ such that
    \begin{equation} \label{b_c_g}
    (b,c)\in g
    \end{equation}
    and
    \begin{equation} \label{c_d_h}
    (c,d)\in h
    \end{equation}
    All this follows directly from the definition of $\circ$.
    By putting equations (\ref{a_b_f}), (\ref{b_c_g}) and (\ref{c_d_h}) together, we see immediately that $(a,d)\in h\circ (g\circ f)$.\\
    This shows $(h\circ g)\circ f=h\circ (g\circ f)$.    
\end{example}

The next example introducing a very often appearing structure in category theory.\\
\begin{example}
	\textbf{Monoid}\\
	A monoid is a set $S$ together with a binary relation $\cdot:S\times S\rightarrow S$ fulfilling the following:\\ \\
	(1) it is associative, that is $(x\cdot y)\cdot z = x\cdot (y\cdot z)$\\
	(2) it has a neutral element $e$, which satisfies $e\cdot x=x=x\cdot e$ for all $x\in S$.\\
	
	A moinid can be seen as a category with only one object and morphisms taken to be all elements in $S$. Moreover, composition is defined to be the binary relation $\cdot$.\\ \\
	One easily verifies that all axioms are fulfilled. Note, since the category has only one element, say $A$, any morphism $x$ has $dom(x)=A=cod(x)$. Therefore, composition works between all morphisms.\\ \\
	Examples for monoids are $N, Q, R$, that is, the natural numbers, the rationals and the reals.
	The binary relation can be taken to be the addition or multiplication.
\end{example}

This last example shows very nicely how much morphisms can deviate from being functions between sets. The power of category theory is that everything what holds in $Set$ and is based only on the axioms of categories resp. morphisms, likewise holds in all monoids.

\begin{example}
	\textbf{The category $Mon$}\\
	From monoids we can build a category $Mon$ which objects are all monoids and morphism functions which preserve the monoid structure in the following sense.\\ \\
	If $f:X\rightarrow Y$ is a morphism then \\ 
	(1) for any $x,y\in X$
	$$
	f(x\cdot y)=f(x)\cdot f(y)
	$$
	(2) $f(e_X)=e_Y$ where $e_X, e_Y$ are the corresponding neutral elements of $X$ resp. $Y$ \\ \\
	One easily checks that this still holds when composing two morphisms $f\circ g$. Therefore the composition of functions provides a composition between morphisms which fulfills all axioms of a category.\\	
\end{example}


\end{document}

	
