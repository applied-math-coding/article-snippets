\documentclass[17pt]{extarticle}
%\usepackage[paperheight=4in]{geometry}
\usepackage[top=1cm, bottom=1cm, left=2cm, right=2cm]{geometry}
\pagestyle{empty} %no page numbering
\usepackage[utf8]{inputenc}
\usepackage{graphicx}
\usepackage{amsmath}
\usepackage{amssymb}
\usepackage{amsthm}

\newtheorem{theorem}{Theorem}[section]
\newtheorem{proposition}[theorem]{Proposition}
\newtheorem{lemma}[theorem]{Lemma}
\newtheorem{example}{Example}
\newtheorem{definition}{Definition}
\newtheorem{remark}[theorem]{Remark}
\newtheorem*{theorem*}{Theorem}
\newtheorem*{condition*}{Condition}

\setlength\parindent{0pt} %no indent

\begin{document}
	

\begin{example}
	We take as objects all sets and as morphisms all binary relations between two sets. That is, if $A, B$ are sets and $f:A\rightarrow B$ is a morphism, then $f$ can be viewed as subset of $A\times B$. In general these morphisms are not functions, it could be that $(a,b)\in f$ and $(a, c)\in f$ for $b,c\in B$ with $b\neq c$.\\
	For two morphisms $f:A\rightarrow B$ and $g:B\rightarrow C$ we define composition as follows:
	$$
	(a,c)\in g\circ f \ :\leftrightarrow \ \exists b\in B \ (a,b)\in f \wedge (b,c)\in g 
	$$
	In words, the composition relates $a$ to $c$ if and only if there is a 'bridge' $b$ such that $f$ relates $a$ to $b$ and $g$ relates $b$ to $c$.\\
	Before showing that the axioms of category are fulfilled note how this category is abstracting the notion of functions. Although, from a point-wise view, these morphisms are not functions, they still behave like functions w.r.t an appropriate defined composition.\\
    For any object $A$ we define
    $$id_A =\{(a,a) \ | \ a\in A\}$$
    One immediately sees that this serves correctly as identity morphism.\\
    It remains to proof associativity. Assume we are given $f:A\rightarrow B$, $g:B\rightarrow C$ and $h:C\rightarrow D$.
    $$
    (a,d)\in (h\circ g)\circ f
    $$
    means, there exists $b\in B$ such that
    \begin{equation} \label{a_b_f}
    (a,b)\in f
    \end{equation}
    and
    $$(b,d)\in h\circ g$$
    Moreover, the latter implies there exists $c\in C$ such that
    \begin{equation} \label{b_c_g}
    (b,c)\in g
    \end{equation}
    and
    \begin{equation} \label{c_d_h}
    (c,d)\in h
    \end{equation}
    All this follows directly from the definition of $\circ$.
    By putting equations (\ref{a_b_f}), (\ref{b_c_g}) and (\ref{c_d_h}) together, lets us infer by doing 
    similar considerations that $(a,d)\in h\circ (g\circ f)$.\\
    This shows $(h\circ g)\circ f=h\circ (g\circ f)$.    
\end{example}


\end{document}

	
