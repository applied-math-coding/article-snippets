\documentclass[17pt]{extarticle}
%\usepackage[paperheight=4in]{geometry}
\usepackage[top=1cm, bottom=1cm, left=2cm, right=2cm]{geometry}
\pagestyle{empty} %no page numbering
\usepackage[utf8]{inputenc}
\usepackage{graphicx}
\usepackage{amsmath}
\usepackage{amssymb}
\usepackage{amsthm}

\newtheorem{theorem}{Theorem}
\newtheorem{proposition}[theorem]{Proposition}
\newtheorem{lemma}[theorem]{Lemma}
\newtheorem{example}{Example}
\newtheorem*{example*}{Example}
\newtheorem{definition}{Definition}
\newtheorem{remark}[theorem]{Remark}
\newtheorem*{theorem*}{Theorem}
\newtheorem*{condition*}{Condition}

\setlength\parindent{0pt} %no indent

\begin{document}

\begin{example*}
	Let $X$ be a partial ordered set. That is there exists a binary relation $\leq$ on $X$
	that fulfills:\\
	(1) if $a\leq b$ and $b\leq c$ then $a\leq c$\\
	(2) if $a\leq b$ and $b\leq a$ then $a=b$\\
	(3) $a\leq a$\\ \\
	We can define a category by taking as objects all elements from $X$ and as morphisms all relations of the form $a\leq b$ for any $a,b\in X$.\\
	That means, the set $Hom(a,b)$ is either empty or consists of the only morphism $a\leq b$.\\
	By (1) associativity is trivially satisfied for the composition. Moreover, (3) ensures the existence of the identity map: $id_a = a\leq a$.\\ \\
	For a product $a\times b$ the following necessarily would have to be fulfilled:\\
	For any $x$ with $x\leq a$ and $x\leq b$
	$$x\leq a\times b$$
	Projections must fulfill
	$$a\times b\leq a$$
	and
	$$a\times b\leq b$$
	Altogether, these statements are equivalent of $a\times b$ being the largest lower bound of $a$ and $b$.\\
    This is an interesting example, where the product directly presents a known concept of partial orders.	
\end{example*}




\end{document}

	
