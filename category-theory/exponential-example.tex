\documentclass[17pt]{extarticle}
%\usepackage[paperheight=4in]{geometry}
\usepackage[top=1cm, bottom=1cm, left=2cm, right=2cm]{geometry}
\pagestyle{empty} %no page numbering
\usepackage[utf8]{inputenc}
\usepackage{graphicx}
\usepackage{amsmath}
\usepackage{amssymb}
\usepackage{amsthm}

\newtheorem{theorem}{Theorem}
\newtheorem{proposition}[theorem]{Proposition}
\newtheorem{lemma}[theorem]{Lemma}
\newtheorem{example}{Example}
\newtheorem*{example*}{Example}
\newtheorem{definition}{Definition}
\newtheorem*{definition*}{Definition}
\newtheorem{remark}[theorem]{Remark}
\newtheorem*{theorem*}{Theorem}
\newtheorem*{condition*}{Condition}

\setlength\parindent{0pt} %no indent

\begin{document}
\begin{example*}
	Let us consider the category of all partial ordered sets. This in general is denoted by $\mathbf{Pos}$.\\ \\
	Remember a partial ordered set consists of a binary relation such that for any elements we have: 
	\\
	(1) if $a\leq b$ and $b\leq c$ then $a\leq c$\\
	(2) if $a\leq b$ and $b\leq a$ then $a=b$\\
	(3) $a\leq a$\\ \\
	
	The category $\mathbf{Pos}$ is now formed by taking as objects all partially ordered sets. Morphisms between any two objects $A, B$ in $Pos$, are taken to be all monotone functions, that is,
	$f\in Hom(A, B)$ iff from
	$$a\leq b$$
	follows
	$$f(a)\leq f(b)$$ \\
	\textbf{Exercise}:\\
	Do verify that all axioms of a category are fulfilled by this definitions and hence $Pos$ really is a category.\\ \\
	Next we are going to show that this category contains all products for any pair of objects:\\
	Given two objects $A, B$ in $Pos$, we just take the Cartesian product $A\times B$ as underlying set and define the following binary relation:\\
	$$(a_1,b_1)\leq (a_2, b_2)$$
	iff
	$$a_1\leq a_2$$
	and
	$$b_1\leq b_2$$\\ \\
	\textbf{Exercise}:\\
	Do verify that the above axioms (1), ... (3) are fulfilled for this binary relation and thus making $A\times B$ belonging to $Pos$.\\ \\
	
	As \textbf{projections} we take the usual projections of Cartesian products, that is,
	$$p_A:(a,b)\mapsto a$$
	$$p_B:(a,b)\mapsto b$$
	
	\textbf{Exercise}:\\
	Do verify that $p_A, p_B$ indeed are morphisms in $Pos$ (show that they are monotone).\\ \\
		
	
	Remember the definition of product in category theory. We have to show for two morphisms $f:X\rightarrow A$ and $g:X\rightarrow B$ the existence of a unique morphism $\langle f,g\rangle$ such the $p_A\circ \langle f,g\rangle = f$ resp. $pr_B\circ \langle f,g\rangle = g$.\\ \\
	The function $\langle f,g\rangle$ is given by
	$$x\mapsto (f(x), g(x))$$\\
	The uniqueness can by readable verified together with the definition of the projection maps.\\
	
	\textbf{Exercise}:\\
	Show that $\langle f,g\rangle$ defined as above is a morphism in $Pos$ (show that it is monotone).\\ \\
	
	Next, we show that the category $Pos$ contains the exponential for any pair of objects. This together with the above result implies the $Pos$ is Cartesian closed.\\
	
	We define
	$$C^B:=Hom(B,C)$$
	That is, $C^B$ is just the set of all monotone functions from $B$ to $C$.\\
    The first thing we have to verify is that $C^B$ is in $Pos$. For this we can make $Hom(B,C)$ a partial ordered set by introducing the following binary relation:
    $$g\leq h$$
    iff for all $b\in B$
    $$g(b)\leq h(b)$$
    So, $g\leq h$ iff it is the case evaluated on all elements of $B$.\\ \\
    
    \textbf{Exercise}:\\
    Do verify that the above axioms (1), ... (3) are fulfilled for this binary relation and thus making $C^B$ belonging to $Pos$.\\ \\
    
    Now, we define $\epsilon\in Hom(C^B\times B, C)$ by just being the canonical evaluation, that is,
    $$\epsilon:(g, b)\mapsto g(b)$$\\    
    \textbf{Exercise}:\\
    Show that $\epsilon$ is a morphism in $Pos$ (show it is monotone).\\ \\
	Finally, we define the transpose $\tilde{f}$ of a given morphism $f:A\times B\rightarrow C$ by
	$$\tilde{f}:a\mapsto f(a, \cdot)$$    
	So, $\tilde{f}$ at an element $a\in A$ is obtained from $f$ by fixing the first parameter of $f$ with $a$.\\
	Again, it remains to show that $\tilde{f}$ is a morphism in $Pos$. For this assume $a_1, a_2\in A$ with $a_1\leq a_2$. Then 
	$$\tilde{f}(a_1)=f(a_1, \cdot)$$
	and
	$$\tilde{f}(a_2)=f(a_2, \cdot)$$
	But since for any $b\in B$
	$$f(a_1, b)\leq f(a_2, b)$$
	it follows by definition of the binary relation on $C^B$
	$$\tilde{f}(a_1)\leq \tilde{f}(a_2)$$
	Thus $\tilde{f}$ is monotone and hence a morphism in $Pos$.\\ \\ \\
	From the above definitions it follows
	$$\epsilon\circ (\tilde{f}\times id_B)=f$$
	We can verify this point-wise:
	$$\epsilon\circ (\tilde{f}\times id_B)(a,b)=\epsilon(f(a,\cdot), b)=f(a,b)$$\\
	\textbf{Exercise}:\\
	Verify the $\tilde{f}$ is uniquely determined by requiring $\epsilon\circ (\tilde{f}\times id_B)=f$.\\ \\
	Altogether we have shown that $Pos$ is a Cartesian close category.
\end{example*}
\end{document}

	
