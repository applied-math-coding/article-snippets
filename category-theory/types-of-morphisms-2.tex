\documentclass[17pt]{extarticle}
%\usepackage[paperheight=4in]{geometry}
\usepackage[top=1cm, bottom=1cm, left=2cm, right=2cm]{geometry}
\pagestyle{empty} %no page numbering
\usepackage[utf8]{inputenc}
\usepackage{graphicx}
\usepackage{amsmath}
\usepackage{amssymb}
\usepackage{amsthm}

\newtheorem{theorem}{Theorem}[section]
\newtheorem{proposition}[theorem]{Proposition}
\newtheorem{lemma}[theorem]{Lemma}
\newtheorem*{lemma*}{Lemma}
\newtheorem{example}{Example}
\newtheorem{definition}{Definition}
\newtheorem*{definition*}{Definition}
\newtheorem{remark}[theorem]{Remark}
\newtheorem*{theorem*}{Theorem}
\newtheorem*{condition*}{Condition}

\setlength\parindent{0pt} %no indent

\begin{document}

\begin{definition*}
	\textbf{Monomorphism}\\
	A morphism $f:A\rightarrow B$ of some category is called monomorphism if the following condition holds:\\
	For any two morphisms $g,h:B\rightarrow A$, from $f\circ g=f\circ h$ follows $g=h$.	
\end{definition*}

\begin{definition}
	\textbf{Epimorphism}\\
	A morphism $f:A\rightarrow B$ of some category is called epimorphism if the following condition holds:\\
	For any two morphisms $g,h:B\rightarrow A$, from $g\circ f=h\circ f$ follows $g=h$.	
\end{definition}

These two definitions example resemble the alternative way of describing an injective resp. surjective function.\\

A function $f$ which is injective and surjective has an inverse $f^{-1}$ which fulfills $f\circ f^{-1}=id$ and $f^{-1}\circ f=id$. This can be obtained from $(y,x)\in f^{-1} \ :\leftrightarrow \ (x,y)\in f$. We can ask if the analogous statement holds in general categories. First let us clarify what is meant by inverse in terms of morphisms:

\begin{definition*}
	\textbf{Isomorphism}\\
	A morphism $f:A\rightarrow B$ is called isomorphism if the exists a morphism $f^{-1}:B\rightarrow A$ such that
	$$f^{-1}\circ f=id_A$$
	and
	$$f\circ f^{-1}=id_B$$	
\end{definition*}

The morphism $f^{-1}$ is called \textbf{inverse} and moreover uniquely determined by its definition.\\
To this, assume there is a further inverse $g$.
Then $g\circ f=id_A$. By composing with $f^{-1}$ we obtain $(g\circ f)\circ f^{-1}=id_A\circ f^{-1}$.
Associativity yields, $g\circ id_B=id_A \circ f^{-1}$ and by using properties of the identity, $g=f^{-1}$.\\ \\
In category theory isomorphic objects are usually considered the same. 

\begin{theorem*}
	If $f$ is an isomorphism then it is an epimorphism and monomorphism.
\end{theorem*}
\begin{proof}
	Assume $g$ and $h$ are morphisms that fulfill $f\circ g=f\circ h$. By composing with $f^{-1}$ from the left on
	both sides we obtain $g=h$. This shows $f$ is a monomorphism. In analogous manner one can show that $f$
	is epimorphism.
\end{proof}


\end{document}

	
