\documentclass[17pt]{extarticle}
%\usepackage[paperheight=4in]{geometry}
\usepackage[top=1cm, bottom=1cm, left=2cm, right=2cm]{geometry}
\pagestyle{empty} %no page numbering
\usepackage[utf8]{inputenc}
\usepackage{graphicx}
\usepackage{amsmath}
\usepackage{amssymb}
\usepackage{amsthm}

\newtheorem{theorem}{Theorem}[section]
\newtheorem{proposition}[theorem]{Proposition}
\newtheorem{lemma}[theorem]{Lemma}
\newtheorem*{lemma*}{Lemma}
\newtheorem{example}{Example}
\newtheorem{definition}{Definition}
\newtheorem{remark}[theorem]{Remark}
\newtheorem*{theorem*}{Theorem}
\newtheorem*{condition*}{Condition}

\setlength\parindent{0pt} %no indent

\begin{document}

\begin{definition}
	\textbf{Monomorphism}\\
	A morphism $f:A\rightarrow B$ of some category is call monomorphism if the following condition holds:
	For any two morphisms $g,h:Y\rightarrow X$ from $f\circ g=f\circ h$ follows $g=h$.	
\end{definition}


\begin{definition}
	\textbf{Monomorphism}\\
	A morphism $f:A\rightarrow B$ of some category is call monomorphism if the following condition holds:\\
	For any two morphisms $g,h:B\rightarrow A$, from $f\circ g=f\circ h$ follows $g=h$.	
\end{definition}

\begin{definition}
	\textbf{Epimorphism}\\
	A morphism $f:A\rightarrow B$ of some category is call epimorphism if the following condition holds:\\
	For any two morphisms $g,h:B\rightarrow A$, from $g\circ f=h\circ f$ follows $g=h$.	
\end{definition}

A function $f$ which is injective and surjective has an inverse $f^{-1}$ which fulfills $f\circ f^{-1}=id$ and $f^{-1}\circ f=id$. This can be obtained from $(y,x)\in f^{-1} \ :\leftrightarrow \ (x,y)\in f$. We could ask if the same holds at the category level as well.
First of all let us abstract the term inverse:

\begin{definition}
	\textbf{Isomorphism}\\
	A morphism $f:A\rightarrow B$ is called isomorphism if the exists a morphism $f^{-1}:B\rightarrow A$ such that
	$$f^{-1}\circ f=id_A$$
	and
	$$f\circ f^{-1}=id_B$$	
\end{definition}
This map $f^{-1}$ is called inverse and unique. To this, assume there is a further inverse $g$.
Then $g\circ f=id_A$. By composing with $f^{-1}$ we obtain $(g\circ f)\circ f^{-1}=id_A\circ f^{-1}$.
Associativity yields, $g\circ id_B=id_A f^{-1}$ and by using properties of the identity, $g=f^{-1}$.\\
Note, in most categories isomorphic objects are considered the same. For instance in group theory, two isomorphic groups are typically treated to be the same.

\begin{theorem}
	If $f$ is an isomorphism then it is an epimorphism and monomorphism.
\end{theorem}
\begin{proof}
	Assume $g$ and $h$ are morphisms that fulfill $f\circ g=f\circ h$. By composing with $f^{-1}$ from the left on
	both sides we obtain $g=h$. This shows $f$ is a monomorphism. In analogous manner one can show that $f$
	is epimorphism.
\end{proof}

\begin{example}
	Remember our previous introduced category $Mon$. Let us look at the morphism $i:N\rightarrow Z$
	which is the inclusion of natural numbers into all integers.
	We show $i$ is monomorph. For this let $g,h\in Hom(Z,N)$ and $i\circ g=i\circ h$.
	This implies for any $x\in Z$, $g(x)=h(x)$ and so $g=h$.\\
	Next, we show $i$ is epimorph as well. First we make a simple observation:\\
	For any morphism $f:Z\rightarrow N$ and positive $n\in Z$ we have by definition of morphisms in $Mon$,\\
	$$0=f(n-n)=f(n)+f(-n)$$
	which implies 
	$$-f(n)=f(-n)$$
	Now assume $f\circ i =g\circ i$ for $f,g\in Hom(Z,N)$. This implies trivially $f(z)=g(z)$ for $z\in Z\cap N$.
	For negative $z\in Z$, we have by the previous remark
	$$f\circ i(-z)=f(-z)=-f(z)$$
	and the same of $g$
	$$g\circ i(-z)=-g(z)$$
	Combined this yields $-g(z)=-f(z)$, hence $g(z)=f(z)$.
	Altogether this shows $f=g$ and thus $i$ is epimorph.\\
	Interestingly $i$ is not isomorph! Otherwise there would be a morphism $i^{-1}:Z\rightarrow N$
	such that $i\circ i^{-1}(-1)=-1$, but this is impossible since $-1\notin N$.	
\end{example}

\begin{example}
	This example comes from topology and regards the category $Top$ which objects are topological spaces and the morphisms are taken to be all continuous maps. Consider the identity map $id: (N, \tau)\rightarrow (N,\sigma)$, where $N$ denotes the set of natural numbers one time equipped with the discrete topology $\tau$ and one times with the chaotic topology $\sigma$. The latter consists of the sets $N$ and $\emptyset$ only. Trivially this map is an monomorphism and epimorphism, but it does not have an inverse. This inverse necessarily would be the map  $id: (N, \sigma)\rightarrow (N,\tau)$, but this is not continuous.
\end{example}

These examples show, in the above theorem the other direction in general does not hold.
In category theory, epimorph and monomorph do not imply isomorph.
The question arises, under which additional conditions does this hold?
For this to answer, we need two more definition.

\begin{definition}
	\textbf{Retraction}\\
	A morphism $f\in Hom(A,B)$ is called a retraction if there exists a right inverse that is,
	a $h\in Hom(B,A)$ with
	$$f\circ h=id_B$$
\end{definition}

\begin{definition}
	\textbf{Section}\\
	A morphism $f\in Hom(A,B)$ is called a section  if there exists a left inverse that is,
	a $h\in Hom(B,A)$ with
	$$h\circ f=id_B$$
\end{definition}

With this we can formulate the following theorem which gives answer to the above question.
\begin{theorem}
	A monomorphism $f$ is an isomorphism if and only if it is a retraction.
\end{theorem}
\begin{proof}
	Assume for the monomorphism $f\in Hom(A,B)$ there is $g\in Hom(B,A)$ such that $f\circ g=id_B$.
	Then,
	$$f\circ(g\circ f)=(f\circ g)\circ f=id_B\circ f=f\circ id_B$$
	Since $f$ is monomorph this implies
	$$g\circ f=id_B$$
	and this altogether $g=f^{-1}$.
\end{proof}

In analogy we have
\begin{theorem}
	A epimorphism $f$ is an isomorphism if and only if it is a section.
\end{theorem}
\begin{proof}
	Assume $f\in Hom(A,B)$ is a epimorphism and $g\in Hom(B,A)$ such that $g\circ f=id_B$. Then,
	$$(f\circ g)\circ f =f\circ(g\circ f)=f\circ id_B=id_B\circ f$$
	Since $f$ is epimorph this implies
	$$f\circ g=id_B$$
	and thus $g=f^{-1}$.
\end{proof}

\end{document}

	
