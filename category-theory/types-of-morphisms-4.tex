\documentclass[17pt]{extarticle}
%\usepackage[paperheight=4in]{geometry}
\usepackage[top=1cm, bottom=1cm, left=2cm, right=2cm]{geometry}
\pagestyle{empty} %no page numbering
\usepackage[utf8]{inputenc}
\usepackage{graphicx}
\usepackage{amsmath}
\usepackage{amssymb}
\usepackage{amsthm}

\newtheorem{theorem}{Theorem}[section]
\newtheorem{proposition}[theorem]{Proposition}
\newtheorem{lemma}[theorem]{Lemma}
\newtheorem*{lemma*}{Lemma}
\newtheorem{example}{Example}
\newtheorem{definition}{Definition}
\newtheorem*{definition*}{Definition}
\newtheorem{remark}[theorem]{Remark}
\newtheorem*{theorem*}{Theorem}
\newtheorem*{condition*}{Condition}

\setlength\parindent{0pt} %no indent

\begin{document}

\begin{definition*}
	\textbf{Retraction}\\
	A morphism $f\in Hom(A,B)$ is called a \textbf{retraction} if there exists a right inverse that is,
	a $h\in Hom(B,A)$ with
	$$f\circ h=id_B$$
\end{definition*}

\begin{definition*}
	\textbf{Section}\\
	A morphism $f\in Hom(A,B)$ is called a \textbf{section} if there exists a left inverse that is,
	a $h\in Hom(B,A)$ with
	$$h\circ f=id_B$$
\end{definition*}
\leavevmode\newline
With this we can formulate the following theorem:
\begin{theorem*}
	A monomorphism $f$ is an isomorphism if and only if it is a retraction.
\end{theorem*}
\begin{proof}
	Assume $f\in Hom(A,B)$ is monomorph. Then there is some $g\in Hom(B,A)$ such that 
	$$f\circ g=id_B$$
	Using this, we get
	$$f\circ(g\circ f)=(f\circ g)\circ f=id_B\circ f=f\circ id_B$$
	Since $f$ is monomorph this implies
	$$g\circ f=id_B$$
	and this altogether $g=f^{-1}$.
\end{proof}
\leavevmode\newline
In analogy we have
\begin{theorem*}
	A epimorphism $f$ is an isomorphism if and only if it is a section.
\end{theorem*}
\begin{proof}
	Assume $f\in Hom(A,B)$ is an epimorphism. Then there exists $g\in Hom(B,A)$ such that 
	$$g\circ f=id_B$$
	Using this, we get
	$$(f\circ g)\circ f =f\circ(g\circ f)=f\circ id_B=id_B\circ f$$
	Since $f$ is epimorph this implies
	$$f\circ g=id_B$$
	and thus $g=f^{-1}$.
\end{proof}

\end{document}

	
