\documentclass[17pt]{extarticle}
%\usepackage[paperheight=4in]{geometry}
\usepackage[top=1cm, bottom=1cm, left=2cm, right=2cm]{geometry}
\pagestyle{empty} %no page numbering
\usepackage[utf8]{inputenc}
\usepackage{graphicx}
\usepackage{amsmath}
\usepackage{amssymb}
\usepackage{amsthm}

\newtheorem{theorem}{Theorem}[section]
\newtheorem{proposition}[theorem]{Proposition}
\newtheorem{lemma}[theorem]{Lemma}
\newtheorem{example}{Example}
\newtheorem{definition}{Definition}
\newtheorem{remark}[theorem]{Remark}
\newtheorem*{theorem*}{Theorem}
\newtheorem*{condition*}{Condition}

\setlength\parindent{0pt} %no indent

\begin{document}
We consider all elements $x$ that fulfill the following relation
$$x=x$$
Trivially, all sets fulfill this relation. The question arises, if there exists a set that
contains all these elements. In other words:\\
Is $\{x: \ x=x\}$ a set?\\
If yes, then this set would contain all sets, since each set fulfills $x=x$.
Let us assume it is a set and denote it by $X$. Define
$$y:=\{z\in X \ : \ z\notin z\}$$
By axioms of set-theory, this is a set and moreover either $y\in y$ or $y\notin y$.
If $y\in y$, then since $y\in X$ the definition of $y$ implies $y\notin y$. 
So it must be $y\notin y$. But then, the definition of $y$ implies $y\in y$.\\
All in all, this yields a contradiction and thus $X$ cannot be a set.\\ \\
Although it is not a set, it still makes perfectly sense, in terms of propositional logic, to consider
expressions of the form
$$\{x \ : \ \phi(x)\}$$
where $\phi$ is any formula. Though, we must take in mind that the expression does not always refer to a set. In such cases they are called (proper) classes.


\end{document}

	
