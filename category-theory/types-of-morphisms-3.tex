\documentclass[17pt]{extarticle}
%\usepackage[paperheight=4in]{geometry}
\usepackage[top=1cm, bottom=1cm, left=2cm, right=2cm]{geometry}
\pagestyle{empty} %no page numbering
\usepackage[utf8]{inputenc}
\usepackage{graphicx}
\usepackage{amsmath}
\usepackage{amssymb}
\usepackage{amsthm}

\newtheorem{theorem}{Theorem}[section]
\newtheorem{proposition}[theorem]{Proposition}
\newtheorem{lemma}[theorem]{Lemma}
\newtheorem*{lemma*}{Lemma}
\newtheorem{example}{Example}
\newtheorem{definition}{Definition}
\newtheorem*{definition*}{Definition}
\newtheorem{remark}[theorem]{Remark}
\newtheorem*{theorem*}{Theorem}
\newtheorem*{condition*}{Condition}

\setlength\parindent{0pt} %no indent

\begin{document}

\begin{example}
	Let us view the natural numbers $N$ and all integers $Z$ as objects in the category $Mon$.\\ \\
	The inclusion $i:N\rightarrow Z$ provides a natural morphism within $Mon$ of which we will show that it is monomorph.\\ \\
	For this let $g,h\in Hom(Z,N)$ and $i\circ g=i\circ h$.
	This implies for any $x\in Z$, $g(x)=h(x)$ and so $g=h$.\\ \\	
	Next, we show $i$ is epimorph as well.\\ \\
	For this, let us first make a simple observation:\\
	For any morphism $f:Z\rightarrow N$ and positive $n\in Z$ we have by definition of morphisms in $Mon$,\\
	$$0=f(n-n)=f(n)+f(-n)$$
	which implies 
	$$-f(n)=f(-n)$$\\
	Now assume $f\circ i =g\circ i$ for $f,g\in Hom(Z,N)$. This trivially implies $f(z)=g(z)$ for $z\in Z\cap N$.
	For negative $z\in Z$, we have by the previous remark
	$$f\circ i(-z)=f(-z)=-f(z)$$
	and the same for $g$
	$$g\circ i(-z)=-g(z)$$
	Combined this yields, 
	$$-g(z)=-f(z)$$
	Hence $g(z)=f(z)$.\\
	Altogether this shows $f=g$ and thus $i$ is epimorph.\\ \\
	Interestingly $i$ is \textbf{not isomorph}! Otherwise there would be a morphism $i^{-1}:Z\rightarrow N$
	such that $i\circ i^{-1}(-1)=-1$, but this is impossible since $-1\notin N$.	
\end{example}
\leavevmode\newline
For readers with knowledge in topology:
\begin{example}
	This example comes from topology and regards the category $Top$ which objects are topological spaces and the morphisms are taken to be all continuous maps. Consider the identity map $id: (N, \tau)\rightarrow (N,\sigma)$, where $N$ denotes the set of natural numbers one time equipped with the discrete topology $\tau$ and one times with the chaotic topology $\sigma$. The latter consists of the sets $N$ and $\emptyset$ only. Trivially this map is an monomorphism and epimorphism, but it does not have an inverse. This inverse necessarily would be the map  $id: (N, \sigma)\rightarrow (N,\tau)$, but this is not continuous.
\end{example}


\end{document}

	
