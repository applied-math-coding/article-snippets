\documentclass[17pt]{extarticle}
%\usepackage[paperheight=4in]{geometry}
\usepackage[top=1cm, bottom=1cm, left=2cm, right=2cm]{geometry}
\pagestyle{empty} %no page numbering
\usepackage[utf8]{inputenc}
\usepackage{graphicx}
\usepackage{amsmath}
\usepackage{amssymb}
\usepackage{amsthm}

\newtheorem{theorem}{Theorem}[section]
\newtheorem{proposition}[theorem]{Proposition}
\newtheorem{lemma}[theorem]{Lemma}
\newtheorem{example}{Example}
\newtheorem{definition}{Definition}
\newtheorem{remark}[theorem]{Remark}
\newtheorem*{theorem*}{Theorem}
\newtheorem*{condition*}{Condition}

\setlength\parindent{0pt} %no indent

\begin{document}
	\begin{definition}
		\textbf{Category}\\
A category consists of a class of \textbf{objects} 
$$
A, B, C, \cdots
$$
 and a class of \textbf{morphisms}
 $$f, g, h, \cdots$$
for which the following axioms are fulfilled:\\ \\
(1) To each morphism $f$ there exists operators called \textbf{domain}, \textbf{codomain}
and denoted by
$$ dom(f), \ cod(f)$$
Each operator associates an object to $f$.\\ \\

(2) If for two morphisms $f$ and $g$ we have $cod(f)=dom(g)$, then there exists a morphism
named \textbf{composition} of $f$ and $g$ which is denoted by $g\circ f$.
Moreover, we have 
$$dom(g\circ f)=dom(f)$$
$$ cod(g\circ f)=cod(g)$$

(3) For each object $A$ there exists an morphism named \textbf{identity} and denoted by
$$id_A$$ which fulfills $dom(id_A)=A$ and $cod(id_A)=A$.

(4) The composition is required to be \textbf{associative} that is, for any morphisms $f,g,h$ 
$$(f\circ g)\circ h=f\circ(g\circ h)$$
given that the compositions are defined.

(5) Composing a morphism $f$ that has $dom(f)=A$ and $cod(f)=B$ with corresponding identities, fulfills
$$id_B\circ f=f=f\circ id_A$$
\end{definition}
\end{document}

	
