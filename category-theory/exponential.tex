\documentclass[17pt]{extarticle}
%\usepackage[paperheight=4in]{geometry}
\usepackage[top=1cm, bottom=1cm, left=2cm, right=2cm]{geometry}
\pagestyle{empty} %no page numbering
\usepackage[utf8]{inputenc}
\usepackage{graphicx}
\usepackage{amsmath}
\usepackage{amssymb}
\usepackage{amsthm}

\newtheorem{theorem}{Theorem}
\newtheorem{proposition}[theorem]{Proposition}
\newtheorem{lemma}[theorem]{Lemma}
\newtheorem{example}{Example}
\newtheorem*{example*}{Example}
\newtheorem{definition}{Definition}
\newtheorem*{definition*}{Definition}
\newtheorem{remark}[theorem]{Remark}
\newtheorem*{theorem*}{Theorem}
\newtheorem*{condition*}{Condition}

\setlength\parindent{0pt} %no indent

\begin{document}
In the category of sets, given two set $B$ and $C$, the set of all functions from $C$ to $B$ is often denoted by $B^C$. It can be constructed as a subset of $PP(C\times B)$, where $P$ denotes the power-set operator. We now aim to generalize this concept in terms of category theory.\\

\begin{definition*}
In a category $\mathcal{C}$ for two objects $B, C\in\mathcal{C}$ we call an object exponential, denoted by
	$$\mathbf{C^B}$$
	if the following is satisfied:\\
	(1) $C^B\in\mathcal{C}$\\
	(2) there is a morphism $\epsilon\in Hom(C^B\times B, C)$ such that:\\
	For any $f\in Hom(A\times B, C)$ there exists a unique $\tilde{f}\in Hom(A, C^B)$ with $\epsilon\circ(\tilde{f}\times id_B)=f$.
\end{definition*}
\leavevmode\newline
At first glance this definition might look a little bewildering. First of all $C^B$ is required to be just an object. Don't let yourself confuse by its special notation. To understand the meaning of $\epsilon$ consider in $Set$ a function $f:A\times B\rightarrow C$. If we fix one element of $A$, say $a$, then we obtain a function $f_a:B\rightarrow C$. That is
$$f_a(b)=f(a,b)$$
This way 
$$a\mapsto f_a$$
actually is a mapping 
$$A\rightarrow C^B$$
This mapping corresponds exactly to the transpose $\tilde{f}\in Hom(A\in C^B)$.\\
On the other hand, this transpose is connected to the function $f$ by the property that 
$$\tilde{f}(a)(b)=f(a,b)$$
The r.h.s we also can write as
$$\left(\tilde{f}(a),b\right)\mapsto \tilde{f}(a)(b)$$
or by regarding as product
$$(\tilde{f}\times id_B)(a,b)\mapsto \tilde{f}(a)(b)$$
From these expressions we see that this map operates by evaluating the the first component,
which is in $C^B$, at the point given in the second component, which is in $B$.\\
This reveals the sense of $\epsilon$, and it is to considered like an abstract evaluation.\\ \\
Like for products, the exponential must not necessarily exist. We say a category is \textbf{cartesian closed} if for any pair of objects its product and exponential exists.
\end{document}

	
