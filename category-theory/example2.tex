\documentclass[17pt]{extarticle}
%\usepackage[paperheight=4in]{geometry}
\usepackage[top=1cm, bottom=1cm, left=2cm, right=2cm]{geometry}
\pagestyle{empty} %no page numbering
\usepackage[utf8]{inputenc}
\usepackage{graphicx}
\usepackage{amsmath}
\usepackage{amssymb}
\usepackage{amsthm}

\newtheorem{theorem}{Theorem}[section]
\newtheorem{proposition}[theorem]{Proposition}
\newtheorem{lemma}[theorem]{Lemma}
\newtheorem{example}{Example}
\newtheorem{definition}{Definition}
\newtheorem{remark}[theorem]{Remark}
\newtheorem*{theorem*}{Theorem}
\newtheorem*{condition*}{Condition}

\setlength\parindent{0pt} %no indent

\begin{document}


\begin{example}
	\textbf{Monoid}\\
	This is a very often appearing structure in category theory. A monoid is a set $S$ together with a binary relation $\cdot:S\times S\rightarrow S$ fulfilling the following:\\ 
	(1) it is associative, that is $(x\cdot y)\cdot z = x\cdot (y\cdot z)$\\
	(2) it has a neutral element $e$, which satisfies $e\cdot x=x=x\cdot e$ for all $x\in S$.\\
	
	A moinid can be seen as a category with only one object and morphisms taken to be all elements in $S$. Moreover, composition is defined to be the binary relation $\cdot$.
	One easily verifies that all axioms are fulfilled. Note, since the category has only one element, say $A$, any morphism $x$ has $dom(x)=A=cod(x)$. Therefore, composition works among all morphisms, that is in analogy with the operation $\cdot$ between elements of $S$.\\
	Examples for monoids are $N, Q, R$, the natural numbers, the rationals and the reals.
	The binary relation can be taken to be the addition or multiplication.
\end{example}

\begin{example}
	\textbf{The category $Mon$}\\
	From monoids we can build a category $Mon$ which objects are all monoids and morphism functions which preserve the monoid structure in the following sense.\\ \\
	 If $f:X\rightarrow Y$ is a morphism then \\ 
	(1) for any $x,y\in X$
	$$
	f(x\cdot y)=f(x)\cdot f(y)
	$$
	(2) $f(e_X)=e_Y$ where $e_X, e_Y$ are the corresponding neutral elements of $X$ resp. $Y$ \\ \\
	One easily checks that this still holds when composing two morphisms $f\circ g$. Therefore the composition of functions provides a composition between morphisms which fulfills all axioms of a category.\\	
\end{example}

\end{document}

	
