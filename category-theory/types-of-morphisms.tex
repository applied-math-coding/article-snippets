\documentclass[17pt]{extarticle}
%\usepackage[paperheight=4in]{geometry}
\usepackage[top=1cm, bottom=1cm, left=2cm, right=2cm]{geometry}
\pagestyle{empty} %no page numbering
\usepackage[utf8]{inputenc}
\usepackage{graphicx}
\usepackage{amsmath}
\usepackage{amssymb}
\usepackage{amsthm}

\newtheorem{theorem}{Theorem}[section]
\newtheorem{proposition}[theorem]{Proposition}
\newtheorem{lemma}[theorem]{Lemma}
\newtheorem*{lemma*}{Lemma}
\newtheorem{example}{Example}
\newtheorem{definition}{Definition}
\newtheorem{remark}[theorem]{Remark}
\newtheorem*{theorem*}{Theorem}
\newtheorem*{condition*}{Condition}

\setlength\parindent{0pt} %no indent

\begin{document}
From functions we consider the properties of being injective and surjective.
Let us given a function $f:X\rightarrow Y$ between two sets.\\ \\
\textbf{Injectivity} states that for a given $y$ there is at most one element $x$ such that $f(x)=y$.\\ \\
\textbf{Surjectivity} states that for each $y$ there is at least one $x$ such that $f(x)=y$.\\

\begin{lemma*}
A function $f:X\rightarrow Y$ is injective if and only if for any two function $g,h:Y\rightarrow X$, from $f\circ g=f\circ h$ follows $g=h$.
\end{lemma*}
\begin{proof}
If $f$ is injective then this is clear since otherwise one would have a $y \in Y$ with $g(y)\neq h(y)$ but $f(g(y))=f(h(y))$
which contradicts the injectivity. On the other hand, if above statement would hold for each $g,h$ but $f$ would not be injective, then there is $y\in Y$ and $x_1\neq x_2$ with $f(x_1)=f(x_2)=y$. We construct two different functions $g, h:Y\rightarrow X$, where $g$ only takes the constant value $x_1$ and $h$ the constant value $x_2$. It follows $f\circ g=f\circ h$ and by assumption this implies $g=h$ in contradiction to $g\neq h$.
\end{proof}
And similar for surjectivity:
\begin{lemma*}
	A function $f:X\rightarrow Y$ is surjective if and only if
	for any two function $g,h:Y\rightarrow X$, from $g\circ f=h\circ f$ follows $g=h$.
\end{lemma*}
\begin{proof}
If $f$ is surjective than the entire $Y$ is its image. So $g=h$ follows immediately. The other direction follows from noting that if the image of $f$ would be not the entire $Y$, then two functions with $g\circ f=h\circ f$ always could 
be made different on the part of $Y$ which is not in the image of $f$. We silently excluding the trivial cases of either $Y=\emptyset$ or $X$ having only one element.
\end{proof}



\end{document}

	
