\documentclass[17pt]{extarticle}
%\usepackage[paperheight=4in]{geometry}
\usepackage[top=1cm, bottom=1cm, left=2cm, right=2cm]{geometry}
\pagestyle{empty} %no page numbering
\usepackage[utf8]{inputenc}
\usepackage{graphicx}
\usepackage{amsmath}
\usepackage{amssymb}
\usepackage{amsthm}

\newtheorem{theorem}{Theorem}[section]
\newtheorem{proposition}[theorem]{Proposition}
\newtheorem{lemma}[theorem]{Lemma}
\newtheorem{example}{Example}
\newtheorem{definition}{Definition}
\newtheorem{remark}[theorem]{Remark}
\newtheorem*{theorem*}{Theorem}
\newtheorem*{condition*}{Condition}

\setlength\parindent{0pt} %no indent

\begin{document}
\begin{definition}
	\textbf{Functor}\\
	A functor $F$ between to categories $C$ and $D$ consists of two functions.
	One which maps objects from $C$ to objects in $D$ and one which maps morphisms from
	$C$ to morphisms in $D$. Moreover, the following must hold:\\
	(1) for $f\in Hom(A,B)$, $F(f)\in Hom(F(A), F(B))$\\
	(2) $F(id_A)=id_{F(A)}$\\
	(3) $F(g\circ f)=F(g)\circ F(f)$
\end{definition}
Functors defined this way are structure preserving mappings between categories.
Let us consider an example for a functor.

\begin{example}
	Remember the category $N$ and $Z$ as monoids of natural numbers and integers. Both categories consist
	of only one object. So we could let a functor map the object of $N$ map to the object of $Z$. The morphisms in
	both categories are the elements and the composition, in case the monoids are written additive, is the $+$-operator.
	By (3) we have the requirement
	$$F(n+m)=F(n)+F(m)$$
	And by (2)
	$$F(0)=0$$
	Therefore the set of all possible functors between both categories are exactly the set of all linear mappings.
	For instance $F:n\mapsto 2n$.
\end{example}
For readers with knowledge in topology the following example will be a interest.

\begin{example}
	Given a topological space $X$, we define a category $C$ by taking as objects all its elements.
	The morphisms are defined as follows. For two objects $x,y\in X$, the morphism $x\rightarrow y$ is included in
	$Hom(C)$ if and only if 
	$$\forall U \ U\in \mathcal{B}(x) \rightarrow y\in U$$
	That is, if for any open set $U$ that contains $x$ it also contains $y$.\\
	The composition is defined by
	$$(x\rightarrow y)\circ (y\rightarrow z) = x\rightarrow z$$
	We still have to verify that all axioms for this set of morphisms are fufilled.\\
	Clearly we have $x\rightarrow x\in Hom(C)$. Hence for any object $x$, $id_x\in Hom(C)$. 
	Moreover, is $x\rightarrow y$ and $y\rightarrow z$ are in $Hom(C)$, then each open set $U$ containing $x$ must
	contain $y$. But $U$ is also an open set for $y$, so it must contain $z$ as well. This shows, the composition is well-defined. Now, it is trivial to verify the composition is associative.\\
	Let $X, Y$ be topological spaces and $C_X, C_Y$ the corresponding categories like above. If $f: X\rightarrow Y$ is a continuous mapping, then $f$ is a functor between these categories. We verify the axioms for a functor:\\
	(1) requires that $f$ must operate on morphisms like so: 
	$$f:(x\rightarrow y)=f(x)\rightarrow f(y)$$
	Then trivially (2) is fulfilled, since $f(x\rightarrow x)=f(x)\rightarrow f(x)$.
	By the definition of the composition, (3) is seen immediately as well. It remains to show that from $(x\rightarrow y)\in Hom(C_X)$ it follows $(f(x)\rightarrow f(y))\in Hom(C_Y)$.\\
	Let $V$ be an open set containing $f(x)$. Then by continuity $f^{-1}(V)$ is open as well. Moreover, $x\in f^{-1}(V)$ and thus $y\in f^{-1}(V)$. But this implies $f(y)\in V$ that shows $(f(x)\rightarrow f(y))\in Hom(C_Y)$.\\
	The question arises if a functor between these categories necessarily is a continuous function. The answer is no. Just consider $f:[0,1]\rightarrow \{0,1\}$, with $f|[0,1/2[=0$ and $f|[1/2, 1]=1$. Let $\{0,1\}$ carry the discrete topology and $[0,1]$ having a topology generated by the following basis: all open sets for points $\xi\in ]0,1]$ and an open set for $0$ which is ${0,1/4}$. This implies $0\rightarrow 1/4$ and trivially $f(0)\rightarrow f(1/4)$. By construction, $f$ is not continuous at $1/2$.
	
	
\end{example}


\end{document}

	
