\documentclass[17pt]{extarticle}
%\usepackage[paperheight=4in]{geometry}
\usepackage[top=1cm, bottom=1cm, left=2cm, right=2cm]{geometry}
\pagestyle{empty} %no page numbering
\usepackage[utf8]{inputenc}
\usepackage{graphicx}
\usepackage{amsmath}
\usepackage{amssymb}
\usepackage{amsthm}

\newtheorem{theorem}{Theorem}[section]
\newtheorem{proposition}[theorem]{Proposition}
\newtheorem{lemma}[theorem]{Lemma}
\newtheorem{example}{Example}
\newtheorem{definition}{Definition}
\newtheorem{remark}[theorem]{Remark}
\newtheorem*{theorem*}{Theorem}
\newtheorem*{condition*}{Condition}

\setlength\parindent{0pt} %no indent

\begin{document}
\begin{definition}
	\textbf{Functor}\\
	A functor $F$ between to categories $C$ and $D$ consists of two functions.
	One which maps objects from $C$ to objects in $D$ and one which maps morphisms from
	$C$ to morphisms in $D$. Moreover, the following must hold:\\
	(1) for $f\in Hom(A,B)$, $F(f)\in Hom(F(A), F(B))$\\
	(2) $F(id_A)=id_{F(A)}$\\
	(3) $F(g\circ f)=F(g)\circ F(f)$
\end{definition}
Functors defined this way are structure preserving mappings between categories.

TODO example F:N to Z   N, Z as monoids
TODO p 10, 6.
TODO 6.1

\end{document}

	
