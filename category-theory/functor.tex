\documentclass[17pt]{extarticle}
%\usepackage[paperheight=4in]{geometry}
\usepackage[top=1cm, bottom=1cm, left=2cm, right=2cm]{geometry}
\pagestyle{empty} %no page numbering
\usepackage[utf8]{inputenc}
\usepackage{graphicx}
\usepackage{amsmath}
\usepackage{amssymb}
\usepackage{amsthm}

\newtheorem{theorem}{Theorem}[section]
\newtheorem{proposition}[theorem]{Proposition}
\newtheorem{lemma}[theorem]{Lemma}
\newtheorem{example}{Example}
\newtheorem{definition}{Definition}
\newtheorem*{definition*}{Definition}
\newtheorem{remark}[theorem]{Remark}
\newtheorem*{theorem*}{Theorem}
\newtheorem*{condition*}{Condition}

\setlength\parindent{0pt} %no indent

\begin{document}
\begin{definition*}
	\textbf{Functor}\\
	A functor $F$ between to categories $C$ and $D$ consists of two functions.
	One which maps objects from $C$ to objects in $D$ and one which maps morphisms from
	$C$ to morphisms in $D$. Moreover, the following must hold:\\
	(1) for $f\in Hom(A,B)$, $F(f)\in Hom(F(A), F(B))$\\
	(2) $F(id_A)=id_{F(A)}$\\
	(3) $F(g\circ f)=F(g)\circ F(f)$
\end{definition*}
\leavevmode\newline
Functors defined this way are structure preserving mappings between categories.
Let us consider some examples.\\

\begin{example}
	Remember the natural numbers and integers, that is $N$ resp. $Z$, are monoids and thus define categories. These both categories consist of only one object and their morphisms are each element in $N$ resp. $Z$. Having two morphisms $m$ and $n$, composition is defined by $m+n$. The identity morphism is given by the element $0$ and with this associativity trivially is fulfilled.\\ \\
	 
    Next, let $F$ by a functor between these categories. Since both only exists of one object, say $x$ resp. $y$, we have $F: x\mapsto y$. So nothing interesting here.\\
    
    Now, let us consider two morphisms $m,n\in N$.\\    
	By (3) we have the requirement
	$$F(n+m)=F(n)+F(m)$$
	And by (2)
	$$F(0)=0$$\\
	Functions that fulfill these two requirements are exactly the linear maps between $N$ and $Z$.
	Therefore $F$ must have the form $F:n\mapsto a\times n$ for some $a\in N$.
\end{example}
\leavevmode\newline
For readers with knowledge in topology the next example will be of interest:

\begin{example}
	Given a topological space $X$ we define a category $C$ by taking as objects all its elements.
	The morphisms are defined as follows: For two objects $x,y\in X$, the morphism $x\rightarrow y$ is included in
	$Hom(C)$ if and only if 
	$$\forall U \ U\in \mathcal{B}(x) \rightarrow y\in U$$
	That is, if for any open set $U$ that contains $x$ it also contains $y$.\\
	The composition is defined by
	$$(y\rightarrow z)\circ (x\rightarrow y)  = x\rightarrow z$$\\
	Let us verify that indeed all axioms for a category are fufilled:\\
	Clearly we have $x\rightarrow x\in Hom(C)$. Hence for any object $x$, $id_x\in Hom(C)$.\\	
	Moreover, if $x\rightarrow y$ and $y\rightarrow z$ are in $Hom(C)$, then each open set $U$ containing $x$ must
	contain $y$. But $U$ is also an open set for $y$, so it must contain $z$ as well. This shows, the composition is well-defined.\\
	Trivially, this composition is associative.\\ \\
	Let $X, Y$ be topological spaces and $C_X, C_Y$ the corresponding categories of the sort like above. If $f: X\rightarrow Y$ is a continuous mapping, then $f$ is a functor between these categories.\\
	For this to see we verify the axioms of a functor:\\ \\
	(1) Assume $x\rightarrow y$. We have to show $f(x)\rightarrow f(y)$.\\	
	Let $V$ be an open set containing $f(x)$. Then by continuity $f^{-1}(V)$ is open as well. Moreover, $x\in f^{-1}(V)$ and thus $y\in f^{-1}(V)$. But this implies $f(y)\in V$ that shows $(f(x)\rightarrow f(y))\in Hom(C_Y)$.\\ \\
	
	
	(2) is trivially fulfilled, since $f(x\rightarrow x)=f(x)\rightarrow f(x)$.\\ \\
	
	(3) is trivial as well but to familiarize the reader with the abstract notation of category theory, let us write down all the details: Assume $x\rightarrow y$ and $y\rightarrow z$, then
	$$f(y\rightarrow z \circ x\rightarrow y)=f(x\rightarrow z)= f(x)\rightarrow f(z)$$
	By using $f(x)\rightarrow f(y)$ and $f(y)\rightarrow f(z)$ we can rewrite the r.h.s as
	$$f(x)\rightarrow f(z)=f(y)\rightarrow f(z)\circ f(x)\rightarrow f(y)$$
	This finally yields
	$$f(x)\rightarrow f(y)=f(y)\rightarrow f(z)\circ f(x)\rightarrow f(y)$$\\
	
	The question arises if a functor between these categories necessarily is a continuous function. The answer is no.\\
	 Just consider the function $f:[0,1]\rightarrow \{0,1\}$, with $f|[0,1/2[=0$ and $f|[1/2, 1]=1$. Let $\{0,1\}$ carry the discrete topology and $[0,1]$ having a topology generated by the following basis: all open sets for points $\xi\in ]0,1]$ and an open set for $0$ which is $\{0,1/4\}$. This implies $0\rightarrow 1/4$ and trivially $f(0)\rightarrow f(1/4)$. By construction, $f$ is not continuous at $1/2$.
	
	
\end{example}


\end{document}

	
