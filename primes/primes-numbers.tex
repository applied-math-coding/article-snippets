\documentclass[17pt]{extarticle}
%\usepackage[paperheight=4in]{geometry}
\usepackage[top=1cm, bottom=1cm, left=2cm, right=2cm]{geometry}
\pagestyle{empty} %no page numbering
\usepackage[utf8]{inputenc}
\usepackage{graphicx}
\usepackage{amsmath}
\usepackage{amssymb}
\usepackage{amsthm}

\newtheorem{theorem}{Theorem}
\newtheorem{proposition}[theorem]{Proposition}
\newtheorem{lemma}[theorem]{Lemma}
\newtheorem*{lemma*}{Lemma}
\newtheorem{example}{Example}
\newtheorem*{example*}{Example}
\newtheorem{definition}{Definition}
\newtheorem*{definition*}{Definition}
\newtheorem{remark}[theorem]{Remark}
\newtheorem*{theorem*}{Theorem}
\newtheorem*{condition*}{Condition}

\setlength\parindent{0pt} %no indent

\begin{document}
One of the most fundamental theorems in number theory is the famous Euclid's division lemma:
\begin{theorem} \label{euclidian_division}
	Let $q\in\mathbb{N}$. Each natural number $n$ has a unique representation of the form
	$$n=m\cdot q+r$$
	with $m, r\geq 0$ and $r<q$.
\end{theorem}
\begin{proof}
	For the cases $q=1$, $n=1$ resp. $q>1$, $n=1$ the unique solutions with $m=1$, $r=0$ 
	resp. $m=0$, $r=1$ are implied.\\
	We first proof the existence of the representation by induction over $n$. So let us assume for some $r<q$ we have
	$$n=m\cdot q+r$$
	Then $n+1$ can be written as
	$$n+1=m\cdot q+r$$
	If $r+1<q$ we are done. If $r+1=q$, we can write
	$$n+1=(m+1)\cdot q$$
	For proving uniqueness, assume
	$$n=m_1\cdot q+r_1=m_2\cdot q+r_2$$
	Without loss of generality assume $m_2>m_1$. Then 
	$$(m_2-m_1)\cdot q + r_2-r_1=0$$
	This yields,
	$$q\leq (m_2-m_1)\cdot q = r_1-r_2$$
	which results in the contradiction $r_1\geq q$.
	We conclude, $m_1=m_2$. From
	$$m_1\cdot q+r_1=m_1\cdot q+r_2$$
	we finally see $r_1=r_2$.	
\end{proof}
$ $\newline
\begin{lemma}
	Let $a,b,p\in \mathbb{N}$ with $p$ being a prime number. The product $a\cdot b$ is divisible by $p$ if and only if either $a$ or $b$ is divisible by $p$.
\end{lemma}
\begin{proof}
	If either $a$ or $b$ is divisible by $p$ it is clear that then its product is divisible by $p$ as well.\\
	In case $a=b=1$ the statement is trivially true. We proceed by induction on the value of each factor, that is,
	we assume the statement is true for all numbers $a', b'$ with $a'<a$ and $b'<b$.\\
	Assume $a\cdot b$ is divisible by $p$ but none of $a$ or $b$ is. Then for some unique positive $r_1$, $r_2$
	we have
	$$a=m_1\cdot p+r_1$$
	$$b=m_2\cdot p+r_2$$
	with
	\begin{equation} \label{r_1_2_lower_p}
		r_1, r_2<p
	\end{equation}
	This yields,
	$$a\cdot b=(m_1m_2q+r_1m_2+m_1r_2)\cdot p + r_1\cdot r_2$$
	with $r_1\cdot r_2>0$.
	Moreover, by divisibility assumption, there must be $m\in\mathbb{N}$ with
	$$r_1\cdot r_2=m\cdot p$$
	Since $r_1<a$ and $r_2<b$, by inductive assumption either $r_1$ or $r_2$ must be divisible by $p$.
	But this contradicts (\ref{r_1_2_lower_p}) and thus we must drop the assumption of neither $a$ nor $b$ not being divisible by $p$.	
\end{proof}
\pagebreak
\begin{lemma}
	Let $p$ be a prime number. A product of natural numbers is divisible by $p$ if and only if
	at least one of its factors is divisible by $p$.
\end{lemma}
\begin{proof}
	This can be seen by induction on the length of the product and application of the previous lemma.
	For instance, $a\cdot b\cdot c=(a\cdot b)\cdot c$.
\end{proof}
$ $\newline
The previous lemma can be used to proof the following theorem.	
\begin{theorem} \label{prime_decomposition}
	Each natural number $n$ greater than $1$ has a unique prime decomposition.
\end{theorem}
\begin{proof}
	The statement is trivially fulfilled in case if $n$ is any prime number. So let us suppose $n$ is
	not a prime number and the statement is fulfilled by induction for all natural numbers below of $n$.
	Then there are some $a, b\in \mathbb{N}$ such that $n=a\cdot b$ and $a,b>1$. Since obviously $a,b<n$, both have a prime decomposition which leads altogether to a prime decomposition of $n$.\\
	The uniqueness of such a decomposition can be seen as follows:\\
	Assume
	$$n=p_1^{m_1}\cdot p_2^{m_2}\cdots p_i^{m_i}$$
	and
	$$n=q_1^{k_1}\cdot q_2^{k_2}\cdots q_j^{k_j}$$
	where $\{p_1, p_2, \dots, p_i\}$ and $\{q_1, q_2, \dots, q_j\}$ are sets of primes.
	Without loss of generality let us assume that $p_1\notin \{q_1, q_2, \dots, q_j\}$, then
	since $n$ is divisible by $p_1$, the product $q_1^{k_1}\cdot q_2^{k_2}\cdots q_j^{k_j}$ must it be
	as well. The lemma implies that at least one factor must be divisible by $p_1$, but this contradicts
	the $q$'s being prime numbers. Since $p_1$ has been chosen arbitrarily, this shows 
	$$\{p_1, p_2, \dots, p_i\}=\{q_1, q_2, \dots, q_j\}$$
	Therefore we may assume to have two decompositions of the form
	$$p_1^{m_1}\cdot p_2^{m_2}\cdots p_i^{m_i}=p_1^{k_1}\cdot p_2^{k_2}\cdots p_i^{k_i}$$
	By repeatedly applying the lemma we observe the l.h.s must be divisible $k_1$ times by $p_1$
	and that the only factor allowing this is $p_1^{m_1}$. This implies $m_1\geq k_1$. The same argumentation we can do with the remaining prime numbers and moreover by interchanging the role of the l.h.s and r.h.s. This finally yields $m_1=k_1, \dots, m_i=k_i$.
\end{proof}
$ $\newline
\begin{theorem}
	There are infinitely many prime numbers.
\end{theorem}
\begin{proof}
	To the contrary assume the set of primes is finite and given by $P:=\{p_1, p_2, \dots, p_n\}$.
	Consider the number 
	$$q:=p_1\cdot p_2 \cdots p_n + 1$$
	By theorem \ref{euclidian_division}, $q$ is not divisible by any of the $p_i$'s. Nor can $q$ be a prime number since it would be greater than all of the $p_i$'s. By theorem \ref{prime_decomposition} there must exist some prime number $p$ that divides $q$. As mentioned $p\notin P$ in contradiction to the assumption. So we conclude, there must exist infinite many primes.
\end{proof}
$ $\newline
\begin{theorem}
	The prime decomposition of a non-prime natural number $n$ with $n>1$ contains a prime number $p$ with
	$$p\leq \lfloor \sqrt{n} \rfloor$$
\end{theorem}
\begin{proof}
	Assume $n=p_1\cdot p_2 \cdots p_m$ being a prime decomposition with possible repetitions in
	the prime number $[p_1, \dots, p_m]$. There cannot be two factors $p_i, p_j$ with 
	$$p_i, p_j > \lfloor \sqrt{n} \rfloor$$
	since otherwise $p_i\cdot p_j > n$.
\end{proof}
$ $\newline
\begin{theorem}
Each prime number greater than $3$ is of the form either $6n-1$ or $6n+1$.	
\end{theorem}
\begin{proof}
	Just note, $6n+2$, $6n+4$ are divisible by $2$ and $6n+3$ by $3$.
	So, only $6n+1$ or $6n+5$ possibly can be prime.
\end{proof}


\end{document}

	
