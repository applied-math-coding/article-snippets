\documentclass[17pt]{extarticle}
%\usepackage[paperheight=3in]{geometry}
\pagestyle{empty} %no page numbering
\usepackage[utf8]{inputenc}
\usepackage{graphicx}
\usepackage{amsmath}
\usepackage{amssymb}

\newtheorem{theorem}{Theorem}[section]
\newtheorem{proposition}[theorem]{Proposition}
\newtheorem{lemma}[theorem]{Lemma}
\newtheorem{example}[theorem]{Example}
\newtheorem{definition}[theorem]{Definition}
\newtheorem{remark}[theorem]{Remark}
\newtheorem{theorem*}{Theorem}

\setlength\parindent{0pt} %no indent

\begin{document}
By the Cauchy-Schwarz inequality we have
\begin{equation} \label{cauchy_schwarz}
|g'(x)|=|\nabla f(x)\cdot h|\leq |\nabla f(x)||h|=|\nabla f(x)|
\end{equation}
This shows, the maximal absolute value of our optimization is restricted by $|\nabla f(x)|$.\\
But on the contrary if we set $h:=\nabla f(x) / |\nabla f(x)|$ then we see
$$
|\nabla f(x)\cdot h|=\left|\nabla f(x)\cdot \frac{\nabla f(x)}{|\nabla f(x)|}\right|= \frac{|\nabla f(x)|^2}{|\nabla f(x)|}=|\nabla f(x)|
$$
This shows, we can attain the maximum in (\ref{cauchy_schwarz}) with this specific value for $h$.\\
But by knowing how to maximize the absolute value, provides us with the $h$ which minimizes the value itself. In detail, by setting $h=-\nabla f(x) / |\nabla f(x)|$ we compute
$$
g'(0)=-\nabla f(x)\cdot \frac{\nabla f(x)}{|\nabla f(x)|}=
- \frac{|\nabla f(x)|^2}{|\nabla f(x)|}=-|\nabla f(x)|
$$
The r.h.s is the smallest value which can be achieved, otherwise it would contradict above found maximal absolute value.\\
We keep this all as a result:
\begin{theorem*}
	The negative gradient is the direction of steepest fall.
\end{theorem*}
\end{document}