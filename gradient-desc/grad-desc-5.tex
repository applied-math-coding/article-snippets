\documentclass[17pt]{extarticle}
\usepackage[top=1cm]{geometry}
\pagestyle{empty} %no page numbering
\usepackage[utf8]{inputenc}
\usepackage{graphicx}
\usepackage{amsmath}
\usepackage{amssymb}
\usepackage{amsthm}

\newtheorem{theorem}{Theorem}[section]
\newtheorem{proposition}[theorem]{Proposition}
\newtheorem{lemma}[theorem]{Lemma}
\newtheorem{example}[theorem]{Example}
\newtheorem{definition}[theorem]{Definition}
\newtheorem{remark}[theorem]{Remark}
\newtheorem{theorem*}{Theorem}
\newtheorem{algorithm}{Algorithm}
\newtheorem*{algorithm*}{Algorithm}

\setlength\parindent{0pt} %no indent

\begin{document}
Up from now we assume the gradient of $f$ to fulfill as so called Lipschitz condition:\\
For some $L>0$ we assume for all $x,y$ to have
\begin{equation} \label{lipschitz}
	|\nabla f(x)-\nabla f(y)|\leq L|x-y|
\end{equation}
This condition is often fulfilled in practical applications.\\ \\
We aim to show $f(x_{n+1})<f(x_n)$ for certain choice of $\tau$: \\ \\

We denote $h_n:=-\nabla f(x_n)$.
By the mean-value theorem there is $\theta\in [0,1]$ such that
$$
f(x_{n+1})=f(x_n)+\nabla f(x_n+\theta\tau h_n)\cdot \tau h_n
$$
We can add and subtract $\nabla f(x_n)\cdot \tau h_n$:
$$
f(x_{n+1})=f(x_n)+\nabla f(x_n)\cdot \tau h_n
+(\nabla f(x_n+\theta\tau h_n)-\nabla f(x_n))\cdot \tau h_n
$$
We estimate this by using Cauchy-Schwarz inequality:
$$
f(x_{n+1})\leq f(x_n)+\nabla f(x_n)\cdot \tau h_n
+\tau|\nabla f(x_n+\theta\tau h_n)-\nabla f(x_n)||h_n|
$$
We back-insert $h_n:=-\nabla f(x_n)$ and we make use of (\ref{lipschitz}):
$$
f(x_{n+1})\leq f(x_n)-\tau|\nabla f(x_n)|^2
+\tau^2 L|\nabla f(x_n)|^2
$$
By choosing $\tau< 1/L$ we have $-\tau+\tau^2 L<0$, and thus
$$
f(x_{n+1})\leq f(x_n)+(-\tau+\tau^2 L)|\nabla f(x_n)|^2<f(x_n)
$$\\
This was our aim to show.


	
\end{document}