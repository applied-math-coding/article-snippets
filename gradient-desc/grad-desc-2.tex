\documentclass[17pt]{extarticle}
%\usepackage[paperheight=3in]{geometry}
\usepackage[top=2cm]{geometry}
\pagestyle{empty} %no page numbering
\usepackage[utf8]{inputenc}
\usepackage{graphicx}
\usepackage{amsmath}
\usepackage{amssymb}

\newtheorem{theorem}{Theorem}[section]
\newtheorem{proposition}[theorem]{Proposition}
\newtheorem{lemma}[theorem]{Lemma}
\newtheorem{example}[theorem]{Example}
\newtheorem{definition}[theorem]{Definition}
\newtheorem{remark}[theorem]{Remark}
\newtheorem{theorem*}{Theorem}

\setlength\parindent{0pt} %no indent

\begin{document}
Notation: \\
$|\cdot|$ denotes the Euclidean norm in $\mathbb{R}^{n}$\\
$x\cdot y$ is the Euclidean inner product in $\mathbb{R}^{n}$\\
$x^T$ denotes the transpose of $x\in\mathbb{R}^n$\\
$\partial_{i}$ is the partial derivative of the $i$'th coordinate\\
\\
Let us fix $x\in\mathbb{R}^{n}$:
\\
For any $h\in\mathbb{R}^{n}$ with $|h|=1$
we define a function $g:\mathbb{R}\rightarrow\mathbb{R}$ by
$$
t\mapsto f(x+t\cdot h)
$$
That is, $g$ follows $f$ along the line $t\mapsto x+t\cdot h$.\\
As we know, the derivative of a function is determining locally its rate of change. Therefore, we easily can formulate our search for the direction of steepest decrease by \\
\\
\textbf{Find $h$ which minimizes $g'(0)$}\\
\\
Since we are only interest in the direction, we require $|h|=1$.\\
Using the chain rule we can compute $g'$ by
$$
g'(0)=\nabla f(x)\cdot h
$$
The expression $\nabla f(x)$ is called gradient of $f$ at $x$ and is computed by
 $$
 \begin{pmatrix}
 \partial_1 f(x)\\
  \partial_2 f(x)\\
   \vdots\\
 \partial_n f(x)
 \end{pmatrix}
 $$

\end{document}