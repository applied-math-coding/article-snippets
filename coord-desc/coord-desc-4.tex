\documentclass[17pt]{extarticle}
%\usepackage[paperheight=4in]{geometry}
\usepackage[top=1cm, bottom=1cm]{geometry}
\pagestyle{empty} %no page numbering
\usepackage[utf8]{inputenc}
\usepackage{graphicx}
\usepackage{amsmath}
\usepackage{amssymb}

\newtheorem{theorem}{Theorem}[section]
\newtheorem{proposition}[theorem]{Proposition}
\newtheorem{lemma}[theorem]{Lemma}
\newtheorem{example}[theorem]{Example}
\newtheorem{definition}[theorem]{Definition}
\newtheorem{remark}[theorem]{Remark}

\setlength\parindent{0pt} %no indent

\begin{document}
Let $\xi$ be an accumulation point of the sequence $(x^{(i)}_j)$. Assume $f'(\xi)\neq 0$.
By starting the iteration at $\xi$ itself, must provide 
in the first cycle at least one coordinate axis $j$ for which 
$$
\arg \min_{\tau}f(\xi+\tau e_i)<f(\xi)
$$
This is since otherwise $\xi$ would be a local minimum and thus $f'(\xi)=0$.
Let us denote this optimum by $\eta$. So,
$$
f(\eta)<f(\xi)
$$
From the iteration formula we obtain
$$
|x_{j+1}^{(i)}-x_{j}^{(i)}|=|\tau_{j}^{(i)}|
$$
which shows the sequence $(\tau_j^{(i)})$ being bounded as well.
Thus the Bolzano-Weierstrass theorem ensures a convergent sequence. By construction we have for each $\tau\in \mathbb{R}$
$$
f(x_j^{(i)}+\tau_j^{(i)}e_j)\leq f(x_j^{(i)}+\tau e_j)
$$
By taking the limit, this gives:
$$
f(\xi+\bar{\tau}e_j)\leq f(\xi+\tau e_j)
$$
In other words $\bar{\tau}$ solves
$$
\arg \min_{\tau}f(\xi+\tau e_j)
$$
Because of the required uniqueness of this optimum,
we must have 
\begin{equation} \label{eta}
\xi+\bar{\tau} e_j=\eta
\end{equation}
By taking the limit of 
$$
x_{j+1}^{(i)}=x_j^{(i)}+\tau_j^{(i)}e_j
$$
we obtain
$$
\xi=\xi+\bar{\tau}e_j
$$
which together with (\ref{eta}) would imply $\xi=\eta$ and contradicts $f(\eta)<f(\xi)$
as pointed out above.


\end{document}