\documentclass[17pt]{extarticle}
\usepackage[paperheight=3in]{geometry}
\pagestyle{empty} %no page numbering
\usepackage[utf8]{inputenc}
\usepackage{graphicx}
\usepackage{amsmath}
\usepackage{amssymb}

\newtheorem{theorem}{Theorem}[section]
\newtheorem{proposition}[theorem]{Proposition}
\newtheorem{lemma}[theorem]{Lemma}
\newtheorem{example}[theorem]{Example}
\newtheorem{definition}[theorem]{Definition}
\newtheorem{remark}[theorem]{Remark}

\setlength\parindent{0pt} %no indent

\begin{document}
If $f(c)<f(d)$, then $f$ must take a local minimum within $[a,d]$,
otherwise in $[c,b]$.\\
This is a consequence of the Intermediate Value Theorem, which states that 
a continuous function on an interval is taking all its intermediate values. 
\end{document}