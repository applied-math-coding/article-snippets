\documentclass[17pt]{extarticle}
%\usepackage[paperheight=4in]{geometry}
\usepackage[top=1cm, bottom=1cm, left=2cm, right=2cm]{geometry}
\pagestyle{empty} %no page numbering
\usepackage[utf8]{inputenc}
\usepackage{graphicx}
\usepackage{amsmath}
\usepackage{amssymb}
\usepackage{amsthm}

\newtheorem{theorem}{Theorem}
\newtheorem{proposition}[theorem]{Proposition}
\newtheorem{lemma}[theorem]{Lemma}
\newtheorem*{lemma*}{Lemma}
\newtheorem{example}{Example}
\newtheorem*{example*}{Example}
\newtheorem{definition}{Definition}
\newtheorem*{definition*}{Definition}
\newtheorem{remark}[theorem]{Remark}
\newtheorem*{theorem*}{Theorem}
\newtheorem*{condition*}{Condition}

\setlength\parindent{0pt} %no indent

\begin{document}
We will use the following more or less standard notation:
$$X\backslash Y:=\{z\in X : z\notin Y\}$$
For convenience let us restate the relation that holds between the sets:
\begin{equation} \label{main_eq}
	A\times B\cup B\times A=C\times D\cup D\times C
\end{equation}
We will proof the statement by contradiction and therefore assume it does not hold. Then either $A$ or $B$ is not equal to any of $D$ or $C$.
Without loss of generality we can assume this holds for $A$, that is, $A\neq C$ and $A\neq D$.
In case $B$ is the one, the same arguments as follow can be applied in analogy.\\
First, we are going to show that neither $A\subset C$ nor $A\subset D$.
To the contrary we assume $A\subset C$.
The case $A\subset D$ can be treated in full analogy.
The entire chain of arguments is a repeatedly application of the equation (\ref{main_eq}).\\
By assumptions we have
$$C\backslash A\neq\emptyset$$
Now if we apply (\ref{main_eq}) from right to left we obtain
$$C\backslash A\times D\subset B\times A$$
This shows two things:
\begin{equation} \label{CwA_subs_B}
C\backslash A\subset B
\end{equation}
\begin{equation} \label{D_subs_A}
D\subset A
\end{equation}
Moreover, since $A\neq D$,
$$A\backslash D\neq\emptyset$$
Applying (\ref{main_eq}) from left to right on the subsets $A\backslash D$ and $C\backslash A$
gives
$$A\backslash D\times C\backslash A\subset C\times D$$
From this we infer
$$C\backslash A\subset D$$
Altogether we have
$$C\backslash A\times C\backslash A\subset C\times D$$
and this can be used to apply (\ref{main_eq}) from right to left:
$$C\backslash A\times C\backslash A\subset A\times B\cup B\times A$$
This gives an immediate contradiction since non of the factors on the left
intersect with $A$. So we must refute the assumption $A\subset C$.\\

So far we have shown that under our assumption we have 
\begin{equation} \label{AwC_not_empty}
A\backslash C\neq\emptyset
\end{equation}
\begin{equation} \label{AwD_not_empty}
A\backslash D\neq\emptyset
\end{equation}
By (\ref{main_eq}) applied from left to right we see
$$A\subset C\cup D$$
and with this using (\ref{AwC_not_empty}), (\ref{AwD_not_empty})
shows the existence of two distinct $a, a'\in A$ with
\begin{equation} \label{a_not_in_C}
	a\in D\backslash C
\end{equation}
\begin{equation} \label{a'_not_in_D}
	a'\in C\backslash D
\end{equation}
This in particular shows $C\backslash D\neq\emptyset$ and $D\backslash C\neq\emptyset$.\\
Applying (\ref{main_eq}) from right to left shows
$$C\backslash D\times D\backslash C\subset A\times B\cup B\times A$$
that implies 
$$C\backslash D\cap B\neq\emptyset \ \text{or} \ D\backslash C\cap B\neq\emptyset$$
So either there is $b\in D\backslash C\cap B$ or $b'\in C\backslash D\cap B$.
Then either
$$(a, b)\in  D\backslash C\times  D\backslash C$$
or
$$(a', b')\in C\backslash D\times C\backslash D$$
Especially this means
$$(a,b) \notin C\times D\cup D\times C$$
resp.
$$(a', b')\notin C\times D\cup D\times C$$
what contradicts (\ref{main_eq}) and shows we must drop our assumptions.\\ \\
In summary we have shown either $A=C$ or $A=D$ resp. either $B=C$ or $B=D$.
In order to finish the proof we have to consider the case $A=B$.\\
Without loss of generality we assume $A=C$. Then (\ref{main_eq}) becomes
$$C\times B=C\times D\cup D\times C$$
from what we infer $B=D$. A similar argument leads to $B=C$ if we assume $A=D$.\\
Finally, we have shown either $A=C$ and $B=D$ or $A=D$ and $B=C$.
\end{document}

	
