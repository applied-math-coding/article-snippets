\documentclass[17pt]{extarticle}
%\usepackage[paperheight=4in]{geometry}
\usepackage[top=1cm, bottom=1cm, left=2cm, right=2cm]{geometry}
\pagestyle{empty} %no page numbering
\usepackage[utf8]{inputenc}
\usepackage{graphicx}
\usepackage{amsmath}
\usepackage{amssymb}
\usepackage{amsthm}

\newtheorem{theorem}{Theorem}
\newtheorem{proposition}[theorem]{Proposition}
\newtheorem{lemma}[theorem]{Lemma}
\newtheorem*{lemma*}{Lemma}
\newtheorem{example}{Example}
\newtheorem*{example*}{Example}
\newtheorem{definition}{Definition}
\newtheorem*{definition*}{Definition}
\newtheorem{remark}[theorem]{Remark}
\newtheorem*{theorem*}{Theorem}
\newtheorem*{condition*}{Condition}

\setlength\parindent{0pt} %no indent

\begin{document}
An often in analysis used theorem states that in $\mathbb{R}^n$ any sequence contained
in a compact subset has a convergent sub-sequence.
Since in the specific case of $\mathbb{R}^n$ compact sets are exactly those that are closed
and bounded, one has the following alternative formulation:
\begin{theorem} \label{conv_sub_sequ_in_R}
	For any bounded sequence in $\mathbb{R}^n$ there is a convergent sub-sequence.
\end{theorem}
This theorem makes a statement about bounded sequences in a complete normed space, that is, a Banach space (B-space).
The immediate question arises if this theorem as well holds for other B-spaces and in particular
the B-space consisting of all continuous real-value functions on some compact $M\subset\mathbb{R}$ equipped with the supremum norm:
$$|f|:=\sup_{x\in M}|f(x)|$$
This norm often is referred as the norm of uniform continuity.\\ \\
Although what follows can be formulated in much more generality, let us restrict to the case of real-valued 
functions defined on some compact set $M\subset\mathbb{R}$.\\ \\
It turns out that the following property is essential to formulate an analogous version of the theorem:
\begin{definition}
	\textbf{Equicontinuity}\\
	A family $\mathcal{F}$ of functions is called equicontinuous if for any $\epsilon>0$ there exists a 
	$\delta >0$ such that for any $x,y\in M$ with $|x-y|<\delta$ implies for all $f\in\mathcal{F}$
	$$|f(x)-f(y)|<\epsilon$$
\end{definition}
So crucial part in this definition is that $\delta$ can be chosen independently of any $f\in\mathcal{F}$.
In other words, by taking an adequate near point $y$ to $x$, we can ensure that for all functions in $\mathcal{F}$ the images of these points do lie $\epsilon$-near.
\begin{theorem}
	\textbf{Arzela-Ascoli}\\
	For an equicontinuous sequence $(f_n)$ of functions which are bounded in the sense that for some $K>0$
	$$|f_n|<K$$
	for all $n$,
	there exists a uniform convergent sub-sequence.
\end{theorem}
\begin{proof}
	If we consider one single point $x\in M$, then since $|f_n(x)|<K$ by Theorem \ref{conv_sub_sequ_in_R}
	there is a convergent sub-sequence. But this sub-sequence is ensured to only converge for this specific point $x$. Since $M$ is a separable space it contains a countable dense subset 
	$$D=\{x_1, x_2, \dots\}$$
	This is, $D$ is such that for any point $x\in M$ any neighborhood $U_{\epsilon}(x)$ has non-empty intersection with $D$.\\
	We construct by induction a sub-sequence $\sigma$ of $(1, 2, \dots)$  as follows:\\
    We start with $x_1\in D$ and pick a sub-sequence $s^1(n)$ like initially pointed out, such that $f_{s^1(n)}(x_1)$ converges. We define 
    $$\sigma(1):=s^1(1)$$
    So we take the first element of $s^1$ only. Then we move to the next element $x_2\in D$ and again pick a sub-sequence $s^2(n)$ of $s^1(n+1)$ such that $f_{s^2(n)}(x_2)$ converges. We define
    $$\sigma(2):=s^2(1)$$
    by again using the first element but now from $s^2(n)$.
    Essentially here to understand is the shift we do when we pick $s^2(n)$ as sub-sequence from $s^1(n+1)$. This makes $s^2(n)$ being a sub-sequence of $(s^1(2), s^1(3), \dots)$.\\
    We can follow with this inductively to define the entire sequence $\sigma$. The construction ensures
    that $\sigma$ up from index $i$ is a sub-sequence of $s^i$. Hence $f_{\sigma(n)}(x_i)$ converges for all $x_i\in D$.\\
    Further, we will show that $f_{\sigma(n)}$ converges for all $x\in M$ as well.
    For this consider any $x\in M$. We will show that $f_{\sigma(n)}(x)$ is a Cauchy-sequence.
    Without loss of generality and to simplify notations let us assume $\sigma=(1, 2, \dots)$.
    Let us given $\epsilon>0$. Consider $z\in D$ with $|x-z|<\delta$ such that for all $i$
    $$|f_i(x)-f_i(z)|<\frac{\epsilon}{3}$$
    Since at $z$ the sequence $(f_n(z))$ converges, there is $N>0$ such that for all $i,j>N$
    $$|f_i(z)-f_j(z)|<\frac{\epsilon}{3}$$
    By making use of 
    $$f_i(x)-f_j(x)=f_i(x)-f_j(z)+f_j(z)-f_i(z)+f_i(z)-f_j(x)$$
    the triangle inequality leads together with the above to the estimate
    $$|f_i(x)-f_j(x)|\leq |f_i(x)-f_j(z)|+|f_j(z)-f_i(z)|+|f_i(z)-f_j(x)|<\epsilon$$
    This shows $(f_n(z))$ is a Cauchy sequence and hence it converges.
    Note, this argument makes indirectly use of $\mathbb{R}$ being a complete metric space, that is, a space in which every Cauchy sequence converges.\\
    So far we already have reached much with most basic techniques. We were able to construct a sub-sequence of $(f_n)$ that converges point-wise. This step is typical for many similar proofs about convergence.
    Next, we will show that the convergence even is uniform. Here it is the first time where the assumed
    compactness of $M$ comes into play. Let us fix some $\epsilon>0$ and consider some $x\in M$.
    By assumption on equicontinuity we can find $\delta>0$ such that 
    $$|f_{\sigma(i)}(x)-f_{\sigma(j)}(z)|<\frac{\epsilon}{3}$$
    for all $z\in U_{\delta}(x)$. Moreover, there is some $N(x)>0$ such that 
    $$|f_{\sigma(i)}(x)-f_{\sigma(j)}(x)|<\frac{\epsilon}{3}$$
    for all $i,j>N(x)$.
    Both together imply by using the triangle inequality similar like above
    $$|f_{\sigma(i)}(z)-f_{\sigma(j)}(z)|<\epsilon$$
    for all $z\in U_{\epsilon}(x)$ and all $i,j>N(x)$.
    We can do this construction for all $x\in M$ and cover $M$ with the resulting
    $\{U_{\epsilon}(x): x\in M\}$. Because $M$ is compact, there must be a finite sub-cover
    $\{U_{\epsilon}(x_1), U_{\epsilon}(x_2), \dots, U_{\epsilon}(x_n) \}$.
    By defining
    $$N:=\max\{N(x_i): i \leq n\}$$ we have that for all $x\in M$
    and $i,j>N$
    $$|f_{\sigma(i)}(x)-f_{\sigma(j)}(x)|<\epsilon$$
    But this just expresses $f_{\sigma(n)}$ to be an uniform Cauchy sequence.	
\end{proof}


\end{document}

	
