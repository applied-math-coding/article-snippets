\documentclass[17pt]{extarticle}
%\usepackage[paperheight=4in]{geometry}
\usepackage[top=1cm, bottom=1cm, left=2cm, right=2cm]{geometry}
\pagestyle{empty} %no page numbering
\usepackage[utf8]{inputenc}
\usepackage{graphicx}
\usepackage{amsmath}
\usepackage{amssymb}
\usepackage{amsthm}
\usepackage{hyperref}

\newtheorem{theorem}{Theorem}
\newtheorem{proposition}[theorem]{Proposition}
\newtheorem{lemma}[theorem]{Lemma}
\newtheorem*{lemma*}{Lemma}
\newtheorem{example}{Example}
\newtheorem*{example*}{Example}
\newtheorem{definition}{Definition}
\newtheorem*{definition*}{Definition}
\newtheorem{remark}[theorem]{Remark}
\newtheorem*{theorem*}{Theorem}
\newtheorem*{condition*}{Condition}

\setlength\parindent{0pt} %no indent

\begin{document}
\begin{theorem}
	Let $r\in\mathbb{N}$ with $r>1$.
	Every natural number $n$ can be presented uniquely in the form
	$$n=\sum_{i=0}^m a_i r^i$$
	with $a_i<r$ for all $i$.
\end{theorem}
\begin{proof}
	The proof will be done by induction on $n$. For the case $n=0$ the only presentation obviously
	is
	$$0=0\cdot r^0$$
	Now, assume the statement is true for all natural numbers below of $n$.
	First of all, we ask, what is the greatest natural number $m$ such that 
	$$r^{m}\leq n$$
	By taking the logarithm we find,
	$$m=\left\lfloor\frac{\ln n}{\ln r}\right\rfloor$$
	By the \href{https://en.wikipedia.org/wiki/Euclidean_division#Division_theorem}{Euclid's division lemma}
	there are unique $a_m, b_m\in\mathbb{N}$ with $b_m<r^m$ such that
	$$n=a_m r^m+b_m$$
	The definition of $m$ trivially implies $a_m<r$.
	Moreover, since $b_m<r^m$ and thus $b_m<n$ we can apply the induction hypotheses to get a unique presentation
	$$b_m=\sum_{i=0}^p a_i r_i$$
	Noting that obviously $p<m$, by combining the last two equations and taking their uniqueness into account we obtain the desired presentation of $n$.	
\end{proof}
\end{document}

	
