\documentclass[17pt]{extarticle}
%\usepackage[top=1cm]{geometry}
%\usepackage[paperheight=3in]{geometry}
\pagestyle{empty} %no page numbering
\usepackage[utf8]{inputenc}
\usepackage{graphicx}
\usepackage{amsmath}
\usepackage{amssymb}
\usepackage{amsthm}

\newtheorem{theorem}{Theorem}[section]
\newtheorem{proposition}[theorem]{Proposition}
\newtheorem{lemma}[theorem]{Lemma}
\newtheorem{example}[theorem]{Example}
\newtheorem{definition}[theorem]{Definition}
\newtheorem{remark}[theorem]{Remark}
\newtheorem{theorem*}{Theorem}
\newtheorem{algorithm}{Algorithm}
\newtheorem*{algorithm*}{Algorithm}

\setlength\parindent{0pt} %no indent

\begin{document}
	This works witouth Taylor, just be observing each partial derivati:
	
0<=	f(x_i+h)-f(x)=f'(x)h+r(h)
0 <= ... = -f'(x)h+r(h)
r(h)/h -> 0
so  f'(x)>=0 and -f'(x)>= 0
....
	
Let us look at Taylor's theorem as it is valid for a real valued $C^2$-function (two time continuous differentiable):

$$
f(x+h)=f(x)+h^T f'(x)+\frac{1}{2!}h^{T}D^2f(x+\theta h)h
$$
with $\theta$ being a suitable real in $(0,1)$.

\end{document}