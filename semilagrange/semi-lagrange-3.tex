\documentclass[17pt]{extarticle}
%\usepackage[paperheight=4in]{geometry}
\usepackage[top=1cm, bottom=1cm]{geometry}
\pagestyle{empty} %no page numbering
\usepackage[utf8]{inputenc}
\usepackage{graphicx}
\usepackage{amsmath}
\usepackage{amssymb}
\usepackage{amsthm}

\newtheorem{theorem}{Theorem}[section]
\newtheorem{proposition}[theorem]{Proposition}
\newtheorem{lemma}[theorem]{Lemma}
\newtheorem{example}[theorem]{Example}
\newtheorem{definition}[theorem]{Definition}
\newtheorem{remark}[theorem]{Remark}
\newtheorem*{theorem*}{Theorem}
\newtheorem*{condition*}{Condition}

\setlength\parindent{0pt} %no indent

\begin{document}
In many application the r.h.s $F(t,x)$ in addition depends on $u(t)$ itself. Since $u(t)$ is a function $\mathbb{R}^n\rightarrow\mathbb{R}^n$ this makes $F$ an operator on some appropriate Banach-space $B$:
$$
F:[0,T]\times B\rightarrow B
$$
And example of this could be $F(t,u)=\nabla\cdot\nabla u$ which appears as friction force in the Navier-Stokes equation. Then $u$ would be required to be two times differentiable and as a Banach-space one could consider the two times continuous differentiable functions.\\
If $F$ fulfills a Lipschitz-condition of the sort
$$
|F(t,u)-F(t,w)|\leq L|u-w|
$$
for some $L>0$, then a general form of Picard-Lindeloef theorem states that the following problem locally has a unique solution:
\begin{equation} \label{b_space_problem}
	u'(t)=F(t,u)
\end{equation}
with $u(0)=u_0$.
This is an initial value problem for a vector-field having at time $0$ the value $u_0$.
With this solution $u$ we can pose another initial value problem:
\begin{equation} 
	x'(t)=u(t,x(t))
\end{equation}
with $x(0)=x_0$.
The solution $x$ we search is a function $[0,T]\rightarrow\mathbb{R}^n$.
Under suitable assumptions on $B$, for instance if it is the two times continuous differentiable function space, this system, by again applying Picard-Lindeloef, has a unique solution $x$.\\
But then the function $v:t\mapsto u(t,x(t))$ fulfills
\end{document}

	
