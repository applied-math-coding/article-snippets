\documentclass[17pt]{extarticle}
%\usepackage[paperheight=4in]{geometry}
\usepackage[top=1cm, bottom=1cm]{geometry}
\pagestyle{empty} %no page numbering
\usepackage[utf8]{inputenc}
\usepackage{graphicx}
\usepackage{amsmath}
\usepackage{amssymb}
\usepackage{amsthm}

\newtheorem{theorem}{Theorem}[section]
\newtheorem{proposition}[theorem]{Proposition}
\newtheorem{lemma}[theorem]{Lemma}
\newtheorem{example}[theorem]{Example}
\newtheorem{definition}[theorem]{Definition}
\newtheorem{remark}[theorem]{Remark}
\newtheorem*{theorem*}{Theorem}
\newtheorem*{condition*}{Condition}

\setlength\parindent{0pt} %no indent

\begin{document}
For each volume we intend to solve the following system
\begin{align*}
	&v'(t)=F(t,x(t),v(t))\\
	&x'(t)=v(t)\\
	&x(0)=x_0\\
	&v(0)=v_0
\end{align*}
which is an ordinary differential equation. For a volume located at $x_0$ with velocity $v_0$, this equation describes how this specific volume moves in space and which velocity it has at each point in time.\\
Therefore we are actually in the realm of ODE's and can apply whatever scheme we want to solve above equation.
If moreover we assume that $F$ is continuous and fulfills the Lipschitz-conditions
with constant $L>0$
$$
|F(t,x,v)-F(t,y,v)|\leq L|x-y|
$$
$$
|F(t,x,v)-F(t,x,w)|\leq L|v-w|
$$
then by Picard-Lindeloef above equations have locally a unique solution.\\
So, having the initial velocity field $u(0, x)$ for some bounded region in $\mathbb{R}^n$ over which we lay a grid, we can solve for each grid-point $x_i$ above equation by taking $x(0)=x_i$ and $v(0)=u(0, x_i)$. At least, this would make all our volumes moving along their correct trajectories. The problem with this is, there will appear regions for which only a few or even no trajectories are passing after some time has lapsed. For these regions in effect we won't be able to get knowledge of $u(t,x)$.
\end{document}

	
