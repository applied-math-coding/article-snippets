\documentclass[17pt]{extarticle}
\usepackage[paperheight=4.5in]{geometry}
%\usepackage[top=1cm, bottom=1cm]{geometry}
\pagestyle{empty} %no page numbering
\usepackage[utf8]{inputenc}
\usepackage{graphicx}
\usepackage{amsmath}
\usepackage{amssymb}
\usepackage{amsthm}

\newtheorem{theorem}{Theorem}[section]
\newtheorem{proposition}[theorem]{Proposition}
\newtheorem{lemma}[theorem]{Lemma}
\newtheorem{example}[theorem]{Example}
\newtheorem{definition}[theorem]{Definition}
\newtheorem{remark}[theorem]{Remark}
\newtheorem*{theorem*}{Theorem}
\newtheorem*{condition*}{Condition}

\setlength\parindent{0pt} %no indent

\begin{document}
\begin{align*}
	\frac{\partial{u}}{\partial{t}}+(u\cdot\nabla) u=F(t,x, u)
\end{align*}
where $u$ is a function $[0,T]\times\mathbb{R}^n\rightarrow\mathbb{R}^n$ and
$F$ an operator which maps the function $u$ to another function and may depend explicitly on time $t$ and space $x$.
The expression $u\cdot\nabla u$ reads in coordinates as:
$$
\sum_i^n u_i\frac{\partial u_j}{\partial x_i}
$$
This is the 'problematic' non-linearity.
\end{document}