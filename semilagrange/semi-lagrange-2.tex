\documentclass[17pt]{extarticle}
%\usepackage[paperheight=4in]{geometry}
\usepackage[top=1cm, bottom=1cm]{geometry}
\pagestyle{empty} %no page numbering
\usepackage[utf8]{inputenc}
\usepackage{graphicx}
\usepackage{amsmath}
\usepackage{amssymb}
\usepackage{amsthm}

\newtheorem{theorem}{Theorem}[section]
\newtheorem{proposition}[theorem]{Proposition}
\newtheorem{lemma}[theorem]{Lemma}
\newtheorem{example}[theorem]{Example}
\newtheorem{definition}[theorem]{Definition}
\newtheorem{remark}[theorem]{Remark}
\newtheorem*{theorem*}{Theorem}
\newtheorem*{condition*}{Condition}

\setlength\parindent{0pt} %no indent

\begin{document}
	Let us consider a velocity field given as a function 
	$$
	u:[0,T]\times\mathbb{R}^n\rightarrow\mathbb{R}^n
	$$
	and a particle which location is tracked by $x:[0,T]\rightarrow\mathbb{R}^n$.
	So if we want to know the velocity of the particle at time $t$ we must consider $u(t, x(t))$. This actually we can regard as a function $v:[0,T]\rightarrow\mathbb{R}^n$ defined by $v(t)=u(t, x(t))$.\\
	By Newton's law, the change of velocity is proportional to forces acting on a particle. Thus,
	\begin{equation} \label{newton_law}
	\frac{d v(t)}{dt}=F(t,x(t))
	\end{equation}
	Here $F(t, x(t))$ are all forces acting at time $t$ on the particle at $x(t)$.
	This equation is already very expressive, since it tells exactly how each particle in a fluid (for instance), would move forward.\\
	If we replace $v$ by its definition, that is $v(t)=u(t, x(t))$ and apply the chain rule, the r.h.s calculates to
	\begin{equation} \label{main_eq}
	\frac{d v(t)}{dt}=\frac{d u(t, x(t))}{dt}=\frac{\partial u}{\partial t}+
	\left(\frac{d x(t)}{dt}\cdot \nabla \right)u
	\end{equation}
	Since velocity of the particle at $x(t)$ is given by
	$$
	v(t)=\frac{d x(t)}{dt}
	$$
	we find by inserting this into (\ref{main_eq})
	\begin{align*}
	&	\frac{\partial u}{\partial t}+
		\left(\frac{d x(t)}{dt}\cdot \nabla \right)u=\\
	&\frac{\partial u}{\partial t}+
	\left(v(t)\cdot \nabla \right)u=\frac{\partial u}{\partial t}+
	\left(u\cdot \nabla \right)u
	\end{align*}
	Thus the problem (\ref{newton_law}) is transformed into one involving
	$$
	\frac{\partial u}{\partial t}+
	\left(u\cdot \nabla \right)u=F(t,x)
	$$	
\end{document}

	
