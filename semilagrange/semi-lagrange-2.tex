\documentclass[17pt]{extarticle}
%\usepackage[paperheight=4in]{geometry}
\usepackage[top=1cm, bottom=1cm]{geometry}
\pagestyle{empty} %no page numbering
\usepackage[utf8]{inputenc}
\usepackage{graphicx}
\usepackage{amsmath}
\usepackage{amssymb}
\usepackage{amsthm}

\newtheorem{theorem}{Theorem}[section]
\newtheorem{proposition}[theorem]{Proposition}
\newtheorem{lemma}[theorem]{Lemma}
\newtheorem{example}[theorem]{Example}
\newtheorem{definition}[theorem]{Definition}
\newtheorem{remark}[theorem]{Remark}
\newtheorem*{theorem*}{Theorem}
\newtheorem*{condition*}{Condition}

\setlength\parindent{0pt} %no indent

\begin{document}
	We consider a velocity field in a fluid given by the function 
	$$
	u:[0,T]\times\mathbb{R}^n\rightarrow\mathbb{R}^n
	$$
	Moreover, we regard a small volume within this fluid which location shall be tracked by $x:[0,T]\rightarrow\mathbb{R}^n$.
	So if we want to know the velocity of this volume at time $t$ we must consider $u(t, x(t))$. This actually we can regard as a function in time only $v:[0,T]\rightarrow\mathbb{R}^n$ defined by 
	$$
	v(t):=u(t, x(t))
	$$
	By Newton's law, the change of velocity is proportional to forces acting on the volume. Thus,
	\begin{equation} \label{newton_law}
	v'(t)=F(t,x(t), u(t))
	\end{equation}
	Here $F(t, x, u)$ collects all these forces. Without going into more details, the appearance of $u$ in that function is due to friction forces or sometimes referred as momentum diffusion and proportional to $\nabla\cdot\nabla u$.\\
	By replacing $v(t)$ with $u(t, x(t))$ and applying the chain rule we compute
	\begin{equation*}
	v'(t)=u'(t, x(t))=\frac{\partial u}{\partial t}+
	\left(\frac{d x(t)}{dt}\cdot \nabla \right)u
	\end{equation*}
	Since velocity of the volume is given by
	$$
	v(t)=\frac{d x(t)}{dt}
	$$
	we find
	\begin{align*}
	&	\frac{\partial u}{\partial t}+
		\left(\frac{d x(t)}{dt}\cdot \nabla \right)u=\\
	&\frac{\partial u}{\partial t}+
	\left(v(t)\cdot \nabla \right)u=\frac{\partial u}{\partial t}+
	\left(u\cdot \nabla \right)u
	\end{align*}
	By replacing this result at the l.h.s of (\ref{newton_law}) we arrive at
	$$
	\frac{\partial u}{\partial t}+
	\left(u\cdot \nabla \right)u=F(t,x, u)
	$$	
\end{document}

	
